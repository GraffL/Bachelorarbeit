\section{Die Gelfand-Dualität}\label{sec:GD}

Im vorangegangenen Kapitel haben wir gezeigt, dass jede \CAlg{} $\A$ (isometrisch) isomorph zu einem Funktionenraum über einem kompakten Hausdorffraum ist (nämlich der Algebra $\stetig(\SpecC\A)$). 

In diesem Kapitel werden wir nun sehen, dass umgekehrt auch jeder kompakte Hausdorffraum $X$ homöomorph zu einem bestimmten Funktionenraum über  einer \CAlg{} ist (nämlich dem Raum $\SpecC(\stetig(X))$). Damit zeigt sich, dass \CAlgn{} und kompakte Hausdorffräume in gewisser Weise äquivalent zueinander sind, was die Aussage der Gelfand-Dualität sein wird.

\subsection{Der \glqq umgekehrte\grqq{} Satz von Gelfand-Neumark}

Ganz analog zum Satz von Gelfand-Neumark wollen wir zeigen, dass jeder kompakte Hausdorffraum bis auf Homöomorphie der Raum aller stetigen Algebrenhomomorphismen von einer \CAlg{} nach $\CC$ ist.

\begin{lemma}\label{lemma:NG}
Sei $X$ ein kompakter Hausdorffraum, dann ist die Abbildung
	\[\Homoeo: X \to \SpecC(\stetig(X)): x \mapsto (f_x: \tau \mapsto \tau(x))\]
ein Homöomorphismus.
\end{lemma}

\begin{proof}Die wesentlichen Teile dieses Beweises entstammen \cite[Bemerkung 2.1.33]{Baer2003}.

Zunächst ist die Abbildung wohldefiniert, denn sind $x \in X$, $\tau_0 \in \stetig(X)$ und $\epsilon > 0$, so gilt für alle $\tau \in \stetig(X)$ mit $\norm{\tau_0 - \tau} < \epsilon$:
	\[\left|f_x(\tau_0) - f_x(\tau)\right| = \left|\tau_0(x) - \tau(x)\right| \leq \underset{y \in X}{\sup} \left|\tau_0(y) - \tau(y)\right| = \norm{\tau_0 - \tau} < \epsilon \]
Also ist $f_x$ stetig. Aus der Definition der Algebra-Struktur auf $\stetig(X)$ (siehe \cref{lemma:CX}) ergibt sich außerdem direkt, dass $f_x$ ein Algebrenhomomorphismus ist. Insgesamt ist somit tatsächlich $f_x \in \SpecC(\stetig(X))$.

Weiterhin ist $\Homoeo$ stetig, denn sind $x \in X$ und 
	\[U(\Homoeo(x), \tau, \epsilon) := \{g \in \SpecC(\stetig(X)) ~|~ |\Homoeo(x)(\tau) - g(\tau)| < \epsilon\}\]
eine offene Umgebung von $\Homoeo(x)$ (siehe \cref{defn:schwachSternTop}), so gilt für alle $y \in \tau^{-1}(B_\epsilon(x))$ (eine offene Menge, da $\tau \in \stetig(X)$, also stetig):
	\[\left|\Homoeo(x)(\tau) - \Homoeo(y)(\tau)\right| = \left|f_x(\tau) - f_y(\tau)\right| = \left|\tau(x) - \tau(y)\right| < \epsilon \]
Das heißt $\Homoeo(y) \in U(\Homoeo(x), \tau, \epsilon)$ und somit ist $\Homoeo$ stetig.

$\Homoeo$ ist außerdem injektiv, denn sind $x, y \in X$ mit $x \neq y$, so gibt es dem Lemma von Urysohn (\ref{satz:Ury}) zu Folge eine stetige Funktion $\varphi: X \to \CC$ mit $\varphi(x) = 0$ und $\varphi(y) = 1$. Somit ist 
	\[\Homoeo(x)(\varphi) = f_x(\varphi) = \varphi(x) \neq \varphi(y) = f_y(\varphi) = \Homoeo(y)(\varphi)\]
und daher $\Homoeo(x) \neq \Homoeo(y)$, also $\Homoeo$ injektiv.

Somit ist $\Homoeo: X \to \Homoeo(X)$ stetig und injektiv. Ferner ist $X$ kompakt und $\Homoeo(X) \subseteq \SpecC(\stetig(X))$ hausdorffsch (\cref{lemma:MA}). Damit ist $\Homoeo: X \to \Homoeo(X)$ nach \cref{lemma:komHDHomoeo} bereits ein Homöomorphismus.

Schließlich ist $\Homoeo(X) = \SpecC(\stetig(X))$, denn gäbe es ein $g \in \SpecC(\stetig(X)) \backslash \Homoeo(X)$, so wären $\Homoeo(X)$ und $\{g\}$ disjunkte, abgeschlossene Teilmengen in einem kompakten Hausdorffraum. Wiederum erhalten wir durch Anwendung des Lemmas von Urysohn eine stetige Funktion $\psi: \SpecC(\stetig(X)) \to \CC$ mit $\psi|_{\Homoeo(X)} \equiv 0$ und $\psi(g) = 1$. 

Da nach dem Satz von Gelfand-Neumark $\stetig(\SpecC(\stetig(X)))$ und $\stetig(X)$ isomorph sind, gibt es ein $\varphi \in \stetig(X)$ mit $\AlgIso(\varphi) = \psi$, d.h. für alle $x \in X$ gilt:
	\[0 = \psi(\Homoeo(x)) = \psi(f_x) = \AlgIso(\varphi)(f_x) = \tau_\varphi(f_x) = f_x(\varphi) = \varphi(x)\]
Also ist $\varphi \equiv 0$ und damit auch $\psi \equiv \AlgIso(\varphi) \equiv 0$, was ein Wiederspruch zur Definition von $\psi$ ist. Somit war die Annahme falsch und es gilt tatsächlich $\Homoeo(X) = \SpecC(\stetig(X))$.
\end{proof}

\subsection{Grundbegriffe der Kategorientheorie}\label{sec:Kategorientheorie}

Um die Gelfand-Dualität formal aufschreiben zu können, benötigen wir einige Grundbegriffe der Kategorientheorie. In Anlehnung an \cite[S. 2-23]{Pizza2013} und \cite[S. 1-6]{Ambrogio2009} werden diese im Folgenden kurz definiert sowie die im anschließenden Kapitel benötigten Beispiele genannt.

\begin{defn}[Kategorie]
Eine \emph{Kategorie} $\KatC$ besteht aus einer Klasse von Objekten, zu je zwei Objekten $X, Y$ einer Klasse von \emph{Morphismen} $\Hom_\KatC(X,Y)$, einer assoziativen Komposition $\circ$ von \glqq passenden\grqq{} Morphismen und zu jedem Objekt $Y$ einen \emph{Identitätsmorphismus} $\id_Y$ Dabei muss für alle Objekte $X, Y, Z$ und Morphismen $f \in \Hom_\KatC(X,Y), g \in \Hom_\KatC(Y,Z)$ gelten:
	\[g \circ f \in \Hom_\KatC(X,Z),\qquad \id_Y \circ f = f \qquad\text{und}\qquad g \circ \id_Y = g\]
Zwei Objekte $X, Y$ aus $\KatC$ heißen \emph{isomorph}, wenn es Morphismen $f \in \Hom_\KatC(X,Y)$ und $f^{-1} \in\Hom_\KatC(Y,X)$ gibt, sodass
	\[f^{-1} \circ f = \id_X \qquad\text{und}\qquad f \circ f^{-1} = \id_Y\]
gelten. $f$ heißt dann \emph{Isomorphismus}.
\end{defn}

\begin{bsp}
Mit kompakten Hausdorffräumen als Objekten und den stetigen Abbildungen als Morphismen erhalten wir $\KatTop$, die \emph{Kategorie der kompakten Hausdorffräume}. Zwei kompakte Hausdorffräume sind genau dann isomorph im obigen Sinne, wenn es einen Homöomorphismus zwischen ihnen gibt.
\end{bsp}

\begin{bsp}
Nehmen wir als Objekte die \CAlgn{} und als Morphismen die \CAlgHomn{}, so erhalten wir $\KatCAlg$, die \emph{Kategorie der kommutativen \CAlgn{} mit Eins}. Zwei \CAlgn{} sind genau dann isomorph im obigen Sinne, wenn es einen Isomorphismus im Sinne von \Cref{defn:CAlgIso} zwischen ihnen gibt.
\end{bsp}

\begin{bsp}
Ist $\KatC$ eine Kategorie, so ist $\KatC^\op$ die dazu \emph{duale Kategorie}. Diese besteht aus den gleichen Objekten, aber den \glqq umgedrehten\grqq{} Morphismen. Das heißt zu zwei Objekten $X, Y$ aus $\KatC$ bzw. $\KatC^\op$ ist 
	\[\Hom_{\KatC^\op}(X,Y) := \{f^\op ~|~ f \in \Hom_\KatC(Y,X)\}.\]
Die Komposition in $\KatC^\op$ ist dabei definiert als $f^\op \circ g^\op := (g \circ f)^\op$. Somit ist $f^\op$ genau dann ein Isomorphismus in $\KatC^\op$, wenn $f$ ein Isomorphismus in $\KatC$ ist.
\end{bsp}


\begin{defn}[Funktor]
Seien $\KatC, \KatD$ zwei Kategorien. Dann ist ein Funktor $\Funk: \KatC \to \KatD$ eine Vorschrift, die jedem Objekt $X$ aus $\KatC$ ein Objekt $\Funk(X)$ aus $\KatD$ zuordnet und jedem Morphismus $f \in \Hom_\KatC(X,Y)$ einen Morphismus $\Funk(f) \in \Hom_\KatD(\Funk(X), \Funk(Y))$. Dabei muss gelten:
\begin{itemize}
	\item Für alle Objekte $X$ aus $\KatC$ ist: $\Funk(\id_X) = \id_{\Funk(X)}$.
	\item Für alle komponierbaren Morphismen $f, g$ aus $\KatC$ gilt: $\Funk(g\circ f) = \Funk(g)\circ \Funk(f)$
\end{itemize}
\end{defn}

\begin{bsp}\label{bsp:FunktorSpec}
In \Cref{sec:SpecCA} haben wir gesehen wie wir jeder \CAlg{} einen kompakten Hausdorffraum zuordnen können. Dadurch erhalten wir einen Funktor $\SpecC$:
\[ \begin{array}{@{}rrcl@{}}
	\SpecC: 	&\KatCAlg^\op		&\to 		&\KatTop													\\
				&\A					&\mapsto 	&\SpecC(\A)													\\				
				&\underbrace{(h:\B \to \A)^\op}_{\in \Hom_{\KatCAlg^\op}(\A,\B)} 	&\mapsto	
				&\underbrace{(\varphi_h: \SpecC(\A) \to \SpecC(\B): f \mapsto f \circ h)}_{\in \Hom_{\KatTop}(\SpecC(\A),\SpecC(\B))}
\end{array} \]
\end{bsp}

\begin{proof}$\varphi_h$ ist tatsächlich ein Morphismus von $\KatTop$, d h. eine stetige Abbildung. Denn ist 
	\[U(\varphi_h(f_0), b, \epsilon) := \{g \in \SpecC(B) ~|~ |\varphi_h(f_0)(b)-g(b)| < \epsilon\}\]
eine offene Umgebung von $\varphi_h(f_0) \in \SpecC(\B)$, so gilt für alle $f \in U(f_0,h(b),\epsilon)$:
	\[\left|\varphi_h(f_0)(b)-\varphi_h(f)(b)\right| = \left|f_0\circ h(b) - f\circ h(b)\right| = \left|f_0(h(b))-f(h(b))\right| < \epsilon\]
Also $f \in U(\varphi_h(f_0), b, \epsilon)$.

Seien ferner $\A, \B$ und $\C$ \CAlgn{}, $g \in \Hom_{\KatCAlg^\op}(\B,\C)$, $h \in \Hom_{\KatCAlg^\op}(\A,\B)$ sowie $f \in \SpecC(\A)$, dann gilt:
\begin{itemize}
	\item $\SpecC(\id_\A^\op)(f) = \varphi_{\id_\A}(f) = f \circ \id_\A = f =  \id_{\SpecC(\A)}(f)$
	\item $\SpecC(g^\op \circ h^\op)(f) = \SpecC((h \circ g)^\op)(f) = \varphi_{h \circ g}(f) = f \circ h \circ g = \varphi_g(f \circ h) =$ \newline $ = \SpecC(g^\op)(f \circ h) = \SpecC(g^\op)\left(\varphi_h(f)\right) = \left(\SpecC(g^\op)\circ\SpecC(h^\op)\right)(f)$
\end{itemize}
\end{proof}

\begin{bsp}\label{bsp:FunktorC}
Analog können wir aus der Zuordnung von \CAlgn{} zu kompakten Hausdorffräumen aus  \Cref{sec:CX} einen Funktor $\stetig$ konstruieren:
\[ \begin{array}{@{}rrcl@{}}
	\stetig: 	&\KatTop			&\to 		&\KatCAlg^\op													\\
				&X					&\mapsto 	&\stetig(X)													\\				
				&\underbrace{(\varphi:X \to Y)}_{\in \Hom_{\KatTop}(X,Y)} 	&\mapsto	
				&\underbrace{(h_\varphi: \stetig(Y) \to \stetig(X): \tau \mapsto \tau \circ \varphi)^\op}_{\in \Hom_{\KatCAlg^\op}(\stetig(X),\stetig(Y))}
\end{array} \]
\end{bsp}

\begin{proof}Dass $h_\varphi^\op$ tatsächlich ein Morphismus in $\KatCAlg^\op$ ist, d.h. $h_\varphi$ ein \CAlgHom{}, ergibt sich durch Nachrechnen der Axiome direkt aus der Definition der \CAlg-Struktur auf $\stetig(X)$ und $\stetig(Y)$ (siehe \Cref{lemma:CX}).

Außerdem gilt für kompakte Hausdorffräume $X, Y, Z$ sowie $\varphi \in \Hom_\KatTop(X,Y)$, $\psi \in \Hom_\KatTop(Y,Z)$ und $\tau \in \stetig(Z)$:
\begin{itemize}
	\item $h_{\id_Z}(\tau) = \tau \circ \id_Z = \tau = \id_{\stetig(Z)}(\tau)$, also $\stetig(\id_Z) = (h_{\id_Z})^\op = (\id_{\stetig(Z)})^\op$ und
	\item $h_{\psi\circ \varphi}(\tau) = \tau\circ \psi \circ \varphi = h_\varphi(\tau\circ\psi) = (h_\varphi \circ h_\psi)(\tau)$, also 
		\[\stetig(\psi \circ \varphi) = h_{\psi \circ \varphi}^\op = (h_\varphi \circ h_\psi)^\op = h_\psi^\op \circ h_\varphi^\op. \qedhere\]
\end{itemize}
\end{proof}

\begin{defn}[Kategorienäquivalenz]\label{defn:KatAEquiv}
Zwei Kategorien $\KatC, \KatD$ heißen \emph{äquivalent}, kurz $\KatC \simeq \KatD$, wenn es 
\begin{itemize}
	\item zwei Funktoren $\Funk: \KatC \to \KatD$ und $\Gunk:\KatD \to \KatC$ gibt sowie
	\item für jedes Objekt $X$ aus $\KatC$ einem Isomorphismus $F_X \in \Hom_\KatC(X, \Gunk(\Funk(X)))$, sodass
		\[\forall f \in \Hom_\KatC(X,Y): \Gunk(\Funk(f))\circ F_X = F_Y \circ f,\]
	 und
	\item für jedes Objekt $A$ aus $\KatD$ einem Isomorphismus $G_X \in \Hom_\KatD(\Funk(\Gunk(A)), A)$, sodass	
		\[\forall g \in \Hom_\KatD(A,B): g\circ G_A = G_B \circ \Funk(\Gunk(g)).\]
\end{itemize}

Sind $\KatC$ und $\KatD^\op$ äquivalent, so heißen $\KatC$ und $\KatD$ \emph{dual} zueinander.
\end{defn}

\begin{bem}
Üblicherweise wird die Äquivalenz von Kategorien mit Hilfe des Begriffs der natürlichen Transformationen definiert (siehe dazu \cite[Kapitel 4]{Pizza2013} bzw. \cite[Defintionen 7 \& 10]{Ambrogio2009}). Diesen entspricht in obiger Definition die Gesamtheit aller $F_X$ bzw. $G_X$.
\end{bem}

Ist für zwei Kategorien erst einmal gezeigt, dass sie äquivalent sind, so genügt es häufig eine Aussage in einer der beiden zu zeigen und sie gilt dann automatisch auch analog in der anderen Kategorie. Ein Beispiel für eine derartige Aussage ist die Isomorphie von zwei Objekten:

\begin{lemma}\label{kor:KatIso2}
Seien $\KatC, \KatD$ zwei äquivalente Kategorien und $\Funk: \KatC \to \KatD$ und $\Gunk:\KatD \to \KatC$ die beiden Funktoren aus \Cref{defn:KatAEquiv}. Dann sind zwei Objekte $X, Y$ aus $\KatC$ genau dann isomorph, wenn $\Funk(X)$ und $\Funk(Y)$ isomorph in $\KatD$ sind.
\end{lemma}

\begin{proof}\begin{itemize}
\item[\glqq$\Rightarrow$\grqq]Seien zunächst $X$ und $Y$ zwei isomorphe Opjekte aus $\KatC$. Dann gibt es also Morphismen  $f \in \Hom_\KatC(X,Y), f^{-1} \in \Hom_\KatC(Y,X)$, sodass $f^{-1} \circ f = \id_X$ und $f \circ f^{-1} = \id_Y$. Daraus folgt
	\[\Funk(f^{-1})\circ \Funk(f) = \Funk(f^{-1}\circ f) = \Funk(\id_X) = \id_{\Funk(X)}\]
und analog $\Funk(f)\circ \Funk(f^{-1}) = \id_{\Funk(Y)}$. Also sind $\Funk(X)$ und $\Funk(Y)$ isomorph.

\item[\glqq$\Leftarrow$\grqq]Für die Rückrichtung seien $\Funk(X)$ und $\Funk(Y)$ isomorph. Dann sind mit den gleichen Argumenten wie in \glqq$\Rightarrow$\grqq{} auch $\Gunk(\Funk(X))$ und $\Gunk(\Funk(Y))$ isomorph. 

Seien nun $F_X$, $F_Y$ die Isomorphismen aus \Cref{defn:KatAEquiv} und $f \in \Hom_\KatC(\Gunk(\Funk(X)),\Gunk(\Funk(Y)))$ der Isomorphismus zwischen $\Gunk(\Funk(X))$ und $\Gunk(\Funk(Y))$. Dann ist $F_Y^{-1} \circ f \circ F_X \in \Hom_\C(X,Y)$ ebenfalls ein Isomorphismus, denn
	\[(F_Y^{-1} \circ f \circ F_X) \circ (F_X^{-1} \circ f^{-1} \circ F_Y) = F_Y^{-1} \circ f \circ \id_{\Gunk(\Funk(X))} \circ f^{-1} \circ F_Y = \dots = \id_Y\]
und analog $(F_X^{-1} \circ f^{-1} \circ F_Y)\circ(F_Y^{-1} \circ f \circ F_X)  = \id_X$. Somit sind $X$ und $Y$ isomorph.
\end{itemize}\end{proof}


\subsection{Beweis der Gelfand-Dualität}

Mit Hilfe der soeben eingeführten Begriffe können wir nun die Gelfand-Dualität formal fassen:

\begin{satz}[Gelfand-Dualität]\label{satz:GD}
Die Kategorie der kompakten Hausdorffräume $\KatTop$ ist dual zur Kategorie der kommutativen \CAlgn{} mit Eins $\KatCAlg$.
\end{satz}


\begin{proof}Als Funktoren werden wir $\stetig$ und $\SpecC$ aus \cref{bsp:FunktorC} und \ref{bsp:FunktorSpec} verwenden:
	\[\begin{array}{@{}rrcl@{}}
	 			&\KatTop			&\simeq		&\KatCAlg^\op 												\\
	\stetig: 	&X					&\mapsto	&\stetig(X)													\\
				&(\varphi:X \to Y)	&\mapsto	&(h_\varphi: \stetig(Y) \to \stetig(X): \tau \mapsto \tau \circ \varphi)^\op 		\\
	\SpecC:		&\SpecC(\A)													&\mapsfrom	&\A				\\
				&(\varphi_h: \SpecC(\A) \to \SpecC(\B): f \mapsto f \circ h)	&\mapsfrom	&(h: \B \to \A)^\op
	\end{array}\]

\begin{itemize}
	\item	
Aus \cref{lemma:NG} wissen wir, dass für jeden kompakten Hausdorffraum $X$ die Abbildung $\Homoeo_X: X \to \SpecC(\stetig(X)): x \mapsto (f_x: \tau \mapsto \tau(x))$ ein Isomorphismus in der Kategorie $\KatTop$ ist.

Sei nun $\psi \in \Hom_\KatTop(X,Y)$, dann gilt für alle $x \in X$:
	\[\SpecC(\stetig(\psi)) \circ \Homoeo_X(x) = \SpecC((h_\psi)^\op)(f_x) = \varphi _{h_\psi}(f_x) = f_x \circ h_\psi: \tau \mapsto f_x(\tau \circ \psi)\]
und
 	\[\Homoeo_Y \circ \psi(x) = \Homoeo_Y(\psi(x)) = f_{\psi(x)}: \tau \mapsto \tau(\psi(x)) = (\tau \circ \psi)(x) = f_x(\tau \circ \psi).\]
Also ist: 
	\[\SpecC(\stetig(\psi)) \circ \Homoeo_X = \Homoeo_Y \circ \psi\]


\item
Analog dazu kennen wir aus dem Satz von Gelfand-Neumark (\cref{satz:GN}) zu jeder \CAlg{} $\A$ den Isomorphismus $\AlgIso_\A: \A \to \stetig(\SpecC(\A)): a \mapsto (\tau_a: f \mapsto f(a))$. Folglich ist auch $\AlgIso_\A^\op \in \Hom_{\KatCAlg^\op}(\stetig(\SpecC(\A)),\A)$ ein Isomorphismus. 

Ferner gilt für alle $g^\op \in \Hom_{\KatCAlg^\op}(\A,\B)$ (also $g \in \Hom_\KatCAlg(\B,\A)$):
	\[g^\op\circ\AlgIso_\A^\op = (\AlgIso_\A\circ g)^\op \]
und
	\[\AlgIso_\B^\op \circ \stetig(\SpecC(g^\op)) = \AlgIso_\B^\op \circ \stetig(\varphi_g) = \AlgIso_\B^\op \circ (h_{\varphi_g})^\op = (h_{\varphi_g}\circ\AlgIso_\B)^\op\]
Es genügt also $\AlgIso_\A\circ g = h_{\varphi_g}\circ\AlgIso_\B$ zu zeigen.

Sei dazu $b \in \B$, dann ist
	\[\AlgIso_\A\circ g (b) = \AlgIso_\A(g(b)) = \tau_{g(b)}: f \mapsto f(g(b)) = (f \circ g)(b) = \tau_b(f\circ g)\]
und 
	\[h_{\varphi_g}\circ\AlgIso_\B (b) = h_{\varphi_g}(\tau_b) = \tau_b \circ \varphi_g: f \mapsto \tau_b(\varphi_g(f)) = \tau_b(f\circ g).\]
Damit ist $\AlgIso_\A\circ g = h_{\varphi_g}\circ\AlgIso_\B$ gezeigt und somit ist:
	\[g^\op\circ\AlgIso_\A^\op = (\AlgIso_\A\circ g)^\op = (h_{\varphi_g}\circ\AlgIso_\B)^\op = \AlgIso_\B^\op \circ \stetig(\SpecC(g^\op))\]
\end{itemize}
\end{proof}


\subsection{Anwendung (endlich dimensionale \CAlgn)}\label{sec:Anwendung}

Im einführenden Beispiel in \cref{sec:BeispielCn} haben wir gesehen, dass der $\CC^n$ mit komponentenweiser Multiplikation ein Beispiel für eine \CAlg{} ist. Mit Hilfe der Gelfand-Dualität können wir nun zeigen, dass wir mit diesem Beispiel tatsächlich schon \emph{alle} endlich-dimensionalen \CAlgn{} gesehen haben (bis auf Isomorphie).

\begin{lemma}
Ist $\A$ ein n-dimensionale C*-Algebra, dann ist $\A$ isomorph zu $\CC^n$.
\end{lemma}

\begin{proof}Aus \cref{sec:BeispielCn} wissen wir, dass $\SpecC(\CC^n)$ eine $n$-elementige Menge ist. Können wir nun zeigen, dass $\SpecC(\A)$ ebenfalls eine $n$-elementige Menge ist, dann müssen $\SpecC(\CC^n)$ und $\SpecC(\A)$ homöomorph sein. Aus \cref{satz:GD} folgt dann mit \cref{kor:KatIso2} bereits, dass auch $\CC^n$ und $\A$ isomorph sind.

Angenommen es wäre $|\SpecC(\A)| > n$. Dann wähle $h_1, \dots, h_{n+1} \in \SpecC(\A)$ paarweise verschieden. 

Aus dem Lemma von Urysohn (\ref{satz:Ury}) folgt, dass es für je zwei $k, l \in \{1, \dots, n+1\}$ mit $k \neq l$ eine stetige Funktion $\psi_{kl}: \SpecC(\A) \to \CC$ gibt, sodass $\psi_{kl}(h_k) = 1$ und $\psi_{kl}(h_l) = 0$. Wir setzen zusätzlich $\psi_{kk} \equiv 1$ für alle $k$ und definieren die Abbildung:
	\[\psi_k: \SpecC(\A) \to \CC: h \mapsto \prod_{l=1}^{n+1}\psi_{kl}(h)\]
Diese ist als Produkt stetiger Funktionen ebenfalls stetig und es gilt für alle $k$:
	\[\psi_k(h_j) = \begin{cases} 1 &, k=j \\ 0 &, k\neq j \end{cases}\]
Damit sind $\{\psi_1, \dots, \psi_{n+1}\} \subset \stetig(\SpecC(\A))$ linear unabhängig, denn:
	\begin{align*}
							 0 &= \lambda_1\psi_1 + \dots + \lambda_{n+1}\psi_{n+1}, &\text{ für } \lambda_k \in \CC \\
	\Rightarrow \forall k:   0 &= \lambda_1\psi_1(h_k) + \dots + \lambda_{n+1}\psi_{n+1}(h_k) = \lambda_k &
	\end{align*}
Also müsste die Dimension von $\stetig(\SpecC(\A)) \geq n+1$ sein. Das jedoch steht im Widerspruch dazu, dass $\stetig(\SpecC(\A))$ dem Satz von Gelfand-Neumark zufolge isomorph zu $\A$ ist, welches nach Voraussetzung Dimension $n$ hat.

Also ist $m := |\SpecC(\A)| \leq n$ und daher $\SpecC(\A)$ isomorph zu $\SpecC(\CC^m)$. Somit ist $\A$ isomorph zu $\CC^m$ und folglich $m = n$.
\end{proof}