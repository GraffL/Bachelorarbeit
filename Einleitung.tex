\phantomsection
\addcontentsline{toc}{section}{Einleitung}
\section*{Einleitung}

\begin{center}
Was haben die komplexen Zahlen und ein Raum aus einem einzigen Punkt gemeinsam? 
\end{center}

Eine etwas seltsame Frage vielleicht - nichtsdestotrotz wird die folgende Arbeit eine Antwort darauf liefern. Genauer gesagt handelt es sich dabei um den einfachsten Spezialfall des Satzes von Gelfand-Neumark. Dieser stellt nämlich eine solche Verbindung zwischen \CAlgn{} (einer Art Verallgemeinerung der komplexen Zahlen) auf der einen Seite und gewissen topologischen Räumen auf der anderen Seite her. 

Die Aussage des Satzes von Gelfand-Neumark ist, in der hier betrachteten Fassung\footnote{In seiner ursprünglichen 1943 von Israel Gelfand und Mark Neumark bewiesenen Variante (\cite[Theorem 1]{GN1943}) gilt der Satz auch für nicht-kommutative \CAlgn{} (dort als \glqq normed $*$-ring\grqq{} bezeichnet). Wir werden uns im Folgenden aber auf kommutative \CAlgn{} beschränken, was im originalen Artikel dem ersten Lemma entspricht.}, dass jede \CAlg{} auch als der Raum der stetigen, komplexwertigen Funktionen über einem bestimmten kompakten Hausdorffraum gesehen werden kann (bis auf eine längentreue Isomorphie). Das erste Kapitel der folgenden Arbeit widmet sich - einem Vorlesungsskript von Christian Bär (\cite[S. 49-69]{Baer2003}) folgend - der Definition der notwendigen Begriffe und dem Beweis dieses Satzes.

Daran anschließend werden wir der \glqq umgedrehten\grqq{} Fragestellung nachgehen - d.h. untersuchen, ob auch jeder kompakter Hausdorffraum als Funktionenraum über einer \CAlg{} betrachtet werden kann. Tatsächlich ist dies der Fall und führt zur Gelfand-Dualität, um die es im zweiten Kapitel gehen wird. Diese besagt, dass \CAlgn{} und kompakte Hausdorffräume in einem gewissen Sinne äquivalent zueinander sind - oder, genauer gesagt, dual zueinander.

Im dritten Kapitel schließlich werden wir eine besondere Art von kompakten Hausdorffräumen herausgreifen - die topologischen Mannigfaltigkeiten - und versuchen, die Aussagen der Sätze der vorangegangen Kapitel dafür zu erweitern. Insbesondere werden wir uns dabei fragen:

\begin{center}
Wie kann man einer \CAlg{} \glqq ansehen\grqq{}, ob der zu ihr duale kompakte Hausdorffraum eine topologischen Mannigfaltigkeit ist?
\end{center}