\section{Der Satz von Gelfand-Neumark}\label{sec:GN}

\subsection{Definitionen}

In diesem Kapitel werden die zur Formulierung des Satzes von Gelfand-Neumark notwendigen Objekte definiert. Außerdem werden wir zu diesen einige grundlegende Eigenschaften zeigen, die für den späteren Beweis hilfreich sein werden.

\begin{defn}[Banachalgebra]
Eine \emph{kommutative komplexe Banachalgebra mit Eins} ist ein $\CC$-Vektorraum, mit einer kommutativen Multiplikation, einem Einselement und einer submultiplikativen Norm, bezüglich der er vollständig ist.

D.h. eine kommutative komplexe Banachalgebra mit Eins ist ein $\CC$-Vektorraum $(\A, +)$ mit einer Operationen $\cdot: \A \times \A \to \A: (a,b) \mapsto a\cdot b$, sodass
\begin{itemize}
	\item $\forall a,b,c  \in \A: a\cdot(b+c) = a\cdot b + a\cdot c$ (Distributivität)
	\item $\forall a,b \in A, \lambda \in \CC: \lambda(a\cdot b) = (\lambda a)\cdot b$
	\item $\forall a,b,c \in \A: (a\cdot b)\cdot c = a \cdot (b \cdot c)$ (Assoziativität)
	\item $\forall a, b \in \A: a\cdot b = b \cdot a$ (Kommutativität)	
\end{itemize}
einem ausgezeichneten Element $e \in \A \backslash \{0\}$, sodass
\begin{itemize}
	\item $\forall a \in \A: e\cdot a = a$ (Einselement)
\end{itemize}
sowie einer Normabbildung $\norm{.}: \A \to \RR_{\geq0}: a \mapsto \norm{a}$, sodass
\begin{itemize}
	\item $\forall a,b \in \A: \norm{a \cdot b} \leq \norm{a} \cdot \norm{b}$ (Submultiplikativität)
\end{itemize}
und $(\A, \norm{.})$ vollständig ist.

Die \emph{Dimension} einer Banachalgebra ist die Dimension des ihr zugrunde liegenden Vektorraums.
\end{defn}

\begin{bem}
Da im Folgenden ausschließlich kommutative und komplexe Banachalgebren mit Eins betrachtet werden, nennen wir diese ab sofort kurz \emph{Banachalgebren}. Wie für Multiplikationen üblich wird außerdem oft $ab$ statt $a \cdot b$ verwendet werden.
\end{bem}

\begin{bem}
Gibt es zu einem $a \in \A$ ein $b \in \A$ mit $ab = e$, so ist dieses $b$ eindeutig. Denn gäbe es ein $c \in \A$ mit der gleichen Eigenschaft, so wäre:
	\[c = ce = c(ab) = (ac)b = eb = b\]
Wir nennen ein solches $b$ \emph{Inverses} zu $a$ und bezeichnen es mit $a^{-1}$. Die Menge aller invertierbaren Elemente von $\A$ bezeichnen wir mit $\A^\times$.
\end{bem}

\begin{lemma}\label{lemma:BAlg-Eigenschaften}
Ist $\A$ eine Banachalgebra, so sind folgende Abbildungen stetig:
\begin{itemize}
	\item $\A \times \A \to \A: (a,b) \mapsto a+b$
	\item $\A \times \A \to \A: (a,b) \mapsto a\cdot b$	
	\item $\CC \times \A \to \A: (\lambda,a) \mapsto \lambda a$
	\item $\A^\times \to \A^\times: a \mapsto a^{-1}$
\end{itemize}
\end{lemma}

\begin{proof}Wir folgen hier dem Beweis zu \cite[Lemma 2.1.14]{Baer2003}.

Seien $a_0, b_0 \in \A, \lambda_0 \in \CC$ und $\epsilon > 0$. Dann gilt:
\begin{itemize}
\item Für alle $a,b \in \A$ mit $\norm{a-a_0}$, $\norm{b-b_0} < \frac{\epsilon}{2}$ ist
		\[\norm{(a+b) - (a_0+b_0)} \leq \norm{a-a_0} + \norm{b-b_0} < \frac{\epsilon}{2} + \frac{\epsilon}{2} = \epsilon.\]
		Also ist die Addition stetig.
		
\item Für alle $a,b \in \A$ mit $\norm{a-a_0}$, $\norm{b-b_0} < \min\left\lbrace\sqrt{\frac{\epsilon}{3}}, \frac{\epsilon}{3\norm{a_0}}, \frac{\epsilon}{3\norm{b_0}}\right\rbrace$ ist
		\begin{align*}
		\norm{ab - a_0b_0} = \norm{ab -a_0b +a_0b -a_0b_0} = \norm{(a-a_0)b + (b-b_0)a_0} \leq \\
		\leq \norm{a-a_0}\norm{b} + \norm{b-b_0}\norm{a} = \norm{a-a_0}\norm{b-b_0+b_0} + \norm{b-b_0}\norm{a_0} \leq \\
		\leq \norm{a-a_0}\norm{b-b_0} + \norm{a-a_0}\norm{b_0} + \norm{b-b_0}\norm{a_0} < \\
		< \sqrt{\frac{\epsilon}{3}} \sqrt{\frac{\epsilon}{3}} + \frac{\epsilon}{3\norm{b_0}}\norm{b_0} + \frac{\epsilon}{3\norm{a_0}}\norm{a_0} = \frac{\epsilon}{3} + \frac{\epsilon}{3} + \frac{\epsilon}{3} = \epsilon.		
		\end{align*}
		Also ist die Multiplikation stetig.
		
\item Für alle $\lambda \in \CC$ mit $|\lambda - \lambda_0| < \frac{\epsilon}{\norm{e}}$ ist
		\[\norm{\lambda e - \lambda_0 e} = \norm{(\lambda-\lambda_0)e} = |\lambda - \lambda_0| \norm{e} < \epsilon.\]
		Also ist die Skalarmultiplikation $(\lambda,a) \mapsto \lambda a$ als Verknüpfung der stetigen Abbildungen $\lambda \mapsto \lambda e$ und $(\lambda e, a) \mapsto \lambda e \cdot a = \lambda a$ ebenfalls stetig.
		
\item Ist $a_0 \in \A^\times$, so ist für alle $a \in \A^\times$ mit $\norm{a-a_0}~<~\frac{\epsilon}{\norm{a_0^{-1}}^2+\norm{a_0^{-1}}\epsilon}$:
	\begin{align*}
	\norm{a^{-1}-a_0^{-1}} = \norm{a^{-1}a_0a_0^{-1}-a^{-1}aa_0^{-1}} = \norm{a^{-1}(a_0-a)a_0^{-1}} \leq \\
	\norm{a^{-1}}\norm{a-a_0}\norm{a_0^{-1}} = \norm{a^{-1}-a_0^{-1}+a_0^{-1}}\norm{a-a_0}\norm{a_0^{-1}} \leq \\
	\norm{a^{-1}-a_0^{-1}}\norm{a-a_0}\norm{a_0^{-1}} + \norm{a_0^{-1}}\norm{a-a_0}\norm{a_0^{-1}}
	\end{align*}
Also
	\[\norm{a^{-1}-a_0^{-1}}\left(1 - \norm{a-a_0}\norm{a_0^{-1}}\right) \leq \norm{a-a_0}\norm{a_0^{-1}}^2\]
und wegen
	\[1-\norm{a-a_0}\norm{a_0^{-1}} > 1 - \frac{\epsilon}{\norm{a_0^{-1}}^2+\norm{a_0^{-1}}\epsilon} \norm{a_0^{-1}} = \frac{\norm{a_0^{-1}} + \epsilon - \epsilon}{\norm{a_0^{-1}}+\epsilon} > 0\]
folgt
	\[\norm{a^{-1}-a_0^{-1}} \leq \frac{\norm{a-a_0}\norm{a_0^{-1}}^2}{1 - \norm{a-a_0}\norm{a_0^{-1}}}.\]
Außerdem gilt:
	\begin{align*}
	\norm{a-a_0} < \frac{\epsilon}{\norm{a_0^{-1}}^2+\norm{a_0^{-1}}\epsilon} 
	\Rightarrow \norm{a-a_0}\left(\norm{a_0^{-1}}^2+\norm{a_0^{-1}}\epsilon\right) < \epsilon \\ 
	\Rightarrow \norm{a-a_0}\norm{a_0^{-1}}^2 < \epsilon\left(1-\norm{a-a_0}\norm{a_0^{-1}}\right) 
	\Rightarrow \frac{\norm{a-a_0}\norm{a_0^{-1}}^2}{1-\norm{a-a_0}\norm{a_0^{-1}}} < \epsilon
	\end{align*}
Also ist $\norm{a^{-1}-a_0^{-1}}< \epsilon$ und somit die Inversenbildung stetig.

\end{itemize}
\end{proof}

\begin{defn}[C*-Algebra]
Eine \emph{kommutative \CAlg{} mit Eins} ist eine Banachalgebra $(\A, +, \cdot, e, \norm{.})$ zusammen mit einer Involutionsabbildung $^*:\A \to \A$, die verträglich ist mit der Algebra-Struktur und der Norm.

D.h. für die Involutionsabbildung soll gelten:
\begin{itemize}
	\item $\forall a,b \in \A, \lambda \in \CC: (\lambda a + b)^* = \overline{\lambda}a^* + b^*$ (verträglich mit der  Vektorraumstruktur)
	\item $\forall a,b \in \A: (a^*)^* = a, (a\cdot b)^* = b^*\cdot a^*$ (verträglich mit der Multiplikation)
	\item $\forall a \in \A: \norm{a^*\cdot a} = \norm{a}^2$ (verträglich mit der Norm)
\end{itemize}
\end{defn}

\begin{bem}
Auch die kommutative \CAlgn{} mit Eins werden wir ab sofort kurz als \emph{\CAlgn} bezeichnen.
\end{bem}

\begin{kor}\label{lemma:CAlg-Eigenschaften}
Ist $\A$ eine \CAlg{}, so gilt $e^* = e$ und $\norm{e} = 1$
\end{kor}

\begin{proof}%Entspricht Bem2.1.4
Für das Einselement $e$ einer \CAlg{} ist
	\[e^* = ee^* = (ee^*)^{**} = (e^{**}e^*)^* = (ee^*)^* = (e^*)^* = e\]
und
	\[\norm{e}^2 = \norm{e^*e} = \norm{ee} = \norm{e}.\]
Also muss $\norm{e} = 1$ gelten ($\norm{e} \neq 0$, da $e \neq 0$).
\end{proof}


\begin{defn}[\CAlgHom/Isometrie]\label{defn:AlgHom}\label{defn:CAlgHom}\label{defn:CAlgIso}\label{defn:Isometrie}
Seien $\A, \B$ zwei \CAlgn, dann heißt eine Abbildung $h: \A \to \B$ 
\begin{enumerate}
\item\emph{Algebrenhomomorphismus}, wenn gilt
\begin{itemize}
	\item $\forall a,b \in \A, \lambda \in \CC: h(\lambda a + b) = \lambda h(a) + h(b)$ (linear)
	\item $\forall a,b \in \A: h(ab) = h(a)h(b)$ (multiplikativ)
	\item $h(e) = e$ (einserhaltend)
\end{itemize}

\item\emph{{\CAlgHom}}, wenn zusätzlich gilt:
\begin{itemize}
	\item $\forall a \in \A: h(a^*) = \big(h(a)\big)^*$
\end{itemize}

\item\emph{Isomorphismus von \CAlgn}, wenn $h$ außerdem biijektiv ist.

\item\emph{Isometrie} bzw. \emph{isometrisch}, wenn gilt:
\begin{itemize}
	\item $\forall a \in \A: \norm{h(a)} = \norm{a}$
\end{itemize}

\end{enumerate}
\end{defn}

\begin{bem}
Ein \CAlgHom{} ist immer auch stetig (bzgl. der von den Normen der \CAlgn{} induzierten Topologien), ein Isomorphismus von \CAlgn{} ist sogar immer isometrisch. Ein Beweis hierfür findet sich in  \cite[Korollare 2.1.19 \& 2.1.20]{Baer2003}.
\end{bem}


\begin{defn}[kompakter Hausdorffraum]
Ein \emph{kompakter Hausdorffraum} ist eine Menge $X$ zusammen mit einer Topologie auf $X$, sodass gilt:
\begin{itemize}
	\item Sind $U_i \subseteq X, i \in I$ offene Mengen mit $\bigcup_{i\in I} U_i = X$ (dann heißt $(U_i)_{i \in I}$ \emph{Überdeckung} von $X$), so gibt es eine endliche Teilmenge $J \subseteq I$ mit $\bigcup_{i\in J} U_i = X$ (Kompaktheit).
	\item Sind $x \neq y \in X$ zwei verschiedene Punkte in $X$, so gibt es offene Mengen $U,V \subset X$ mit $x \in U, y \in V$ und $U \cap V = \emptyset$ (Hausdorffeigenschaft).
\end{itemize}
\end{defn}

\begin{defn}[Homöomorphismus]
Seien $X, Y$ zwei topologische Räume. Dann heißt eine Abbildung $\varphi: X \to Y$ \emph{Homöomorphismus}, wenn sie bijektiv ist und sowohl $\varphi$ als auch $\varphi^{-1}$ stetig sind.
\end{defn}

\begin{lemma}\label{lemma:komHDHomoeo}
Ist $X$ ein kompakter topologischer Raum, $Y$ ein Hausdorffraum und $\varphi: X \to Y$ eine stetige bijektive Abbildung, so ist auch $\varphi^{-1}$ stetig (also $\varphi$ ein Homöomorphismus).
\end{lemma}

\begin{proof}Sei $U \subseteq X$ offen. Zu zeigen ist, dass dann auch $\varphi(U) \subseteq Y$ offen ist, d.h. jedes $y_0 \in \varphi(U)$ eine offene Umgebung in $\varphi(U)$ hat.

Da $Y$ hausdorffsch ist, gibt es zu jedem $y \in Y \backslash \varphi(U)$ offene disjunkte Mengen $V_y, W_y \subset Y$ mit $y_0 \in V_y$ und $y \in W_y$. Dann ist $(\varphi^{-1}(W_y))_{y \in Y\backslash \varphi(U)}$ eine offene Überdeckung von $X \backslash U$, denn:
\begin{itemize}
	\item Die $W_y$ sind offen und $\varphi$ ist stetig. Also sind auch die $\varphi^{-1}(W_y)$ offen.
	\item Ist $x \in X \backslash U$, so ist, da $\varphi$ bijektiv ist, $y := \varphi(x) \in Y \backslash \varphi(U)$. Daher ist $x \in \varphi^{-1}(W_y)$ für ein $y \in Y \backslash \varphi(U)$. 
\end{itemize}

$X \backslash U$ ist als abgeschlossene Teilmenge einer kompakten Menge selbst kompakt. Folglich gibt es eine endliche Menge $M \subseteq Y \backslash \varphi(U)$, sodass $X \backslash U \subseteq \bigcup_{y \in M}\varphi^{-1}(W_y)$ ist. Also ist $\bigcap_{y \in M}\varphi^{-1}(V_y) \subseteq U$ und daher $\bigcap_{y \in M}V_y \subseteq \varphi(U)$ eine offene Umgebung von $y_0$.
\end{proof}

\subsection{$\CC^n$ als Beispiel einer \CAlg}\label{sec:BeispielCn}

Bevor wir zum eigentlichen Satz von Gelfand-Neumark kommen, werden wir ihn in diesem Abschnitt zunächst am Spezialfall des $\CC^n$ betrachten. Dieser wird mit komponentenweiser Multiplikation, der Maximumsnorm $||.||_\infty$ und der Involutionsabbildung
\[*: 
\left(\begin{matrix}
 c_1 \\ \vdots \\ c_n 
\end{matrix}\right)
\mapsto 
\left(\begin{matrix}
\overline{c_1} \\ \vdots \\ \overline{c_n}
\end{matrix}\right) \qquad (\overline{c_k} \text{ sei die komplexe Konjugation von } c_k  \text{ in } \CC).\]
eine ($n$-dimensionale) \CAlg{}. Die Einheitsvektoren $e_k = \left( \delta_{1k} \  \dots \ \delta_{nk} \right)^T$ bilden dann eine Basis dieser \CAlg{} und $e := (1,\dots,1)^T = \sum_{k=1}^ne_k$ ist ihr Einselement.

Als erstes wollen wir einen kompakten Hausdorffraum zu dieser \CAlg{} konstruieren. Betrachten wir dazu die Menge
	\[\SpecC(\CC^n) := \{f: \CC^n \to \CC ~|~ f \text{ stetiger Algebrenhomomorphismus}\},\]
so gilt für jeden Algebrenhomomorphismus $f \in \SpecC(\CC^n)$:
	\[f(e_k) = f(e_k \cdot e_k) \overset{Alg.hom.}{=} f(e_k) \cdot f(e_k) \in \CC\]
Also $f(e_k) \in \{0, 1\}$. Gleichzeitig gilt aber auch:
	\[1 = f(e) = f\left(\sum_{k=1}^n e_k\right) \overset{Alg.hom.}{=} \sum_{k=1}^n f(e_k)\]
Insgesamt gibt es daher genau ein $k_f \in \{1, \dots, n\}$ mit $f(e_{k_f}) = 1$. Und durch dieses ist $f$ bereits eindeutig festgelegt, denn es gilt:
	\[\forall c \in \CC^n: f(c) = f\left(\sum_{k=1}^n c_k e_k\right) \overset{Alg.hom.}{=} \sum_{k=1}^n c_k f(e_k) = c_{k_f} \]
Einfaches Nachrechnen zeigt, dass die so definierten $n$ verschiedenen Abbildungen tatsächlich stetige Algebrenhomomorphismen sind. Also ist $\SpecC(\CC^n)$ eine Menge aus $n$ Elementen $\{f_1, \dots, f_n\}$. Versehen mit der diskreten Topologie (jede Teilmenge ist eine offene Menge) wird $\SpecC(\CC^n)$ so offensichtlich zu einem kompakter Hausdorffraum.

Betrachten wir auf diesem Raum nun die Menge der stetigen Abbildungen nach $\CC$:
	\[\stetig(\SpecC(\CC^n)) := \{\tau: \SpecC(\CC^n) \to \CC ~|~ \tau \text{ stetig}\}\]
Versehen wir diese auf natürliche Weise mit der Struktur einer Algebra, der Maximumsnorm $\norm{\tau} := \max \{|\tau(f)| ~|~ f \in \SpecC(\CC^n)\}$ und der komplexen Konjugation als Involution, so wird $\stetig(\SpecC(\CC^n))$ eine \CAlg. 

Da $\SpecC(\CC^n)$ aus genau $n$ verschiedenen Elementen besteht, ist eine Funktion $\tau$ auf diesem Raum definiert durch die Werte von $\tau$ auf diesen $n$ Punkten von $\SpecC(\CC^n)$. Außerdem ist aufgrund der diskreten Topologie \emph{jede} Funktion auf $\SpecC(\CC^n)$ auch stetig. Damit ist die folgende Abbildung bijektiv:
	\[\AlgIso: \CC^n  \to \stetig(\SpecC(\CC^n)): \left(\begin{matrix} c_1 \\ \vdots \\ c_n \end{matrix}\right) \mapsto (\tau_c: \SpecC(\CC^n) \to \CC: f_k \mapsto c_k) \]
Es ist jetzt leicht zu zeigen, dass $\AlgIso$ ein isometrischer Isomorphismus ist.

Insgesamt haben wir also einen kompakten Hausdorffraum gefunden (nämlich $\SpecC(\CC^n)$), dessen Funktionenraum isometrisch isomorph zu unserer ursprünglichen \CAlg{} $\CC^n$ ist.

Eine vergleichbare Konstruktion wollen wir im Folgenden auch für allgemeine \CAlgn{} finden. Dazu werden wir als erstes die Konstruktionen von $\SpecC(\CC^n)$ und $\stetig(\SpecC(\CC^n))$ verallgemeinern:

\subsection{$\SpecC(\A)$ als kompakter Hausdorffraum}\label{sec:SpecCA}

Analog zum obigen Beispiel soll auch für eine beliebige \CAlg{} $\A$ die Menge der stetigen Algebrenhomomorphismen von $\A$ nach $\CC$ als zugehörigen topologischen Raum verwendet werden. Da dieser im Allgemeinen nicht endlich sein muss, wäre die diskrete Topologie hierfür jedoch zu fein und der Raum damit nicht mehr kompakt. Stattdessen werden wir die folgende Topologie verwenden:

\begin{defn}[\ssTop]\label{defn:schwachSternTop}
Sei $(A, ||.||)$ ein $\CC$-Vektorraum und $\dual{A} := \{f: A \to \CC ~|~ f \text{ stetig, linear}\}$ der zugehörige Dualraum. Die von der Subbasis
\[ U(f_0, a, \epsilon) := \{f \in \A ~|~ |f_0(a)-f(a)| < \epsilon\} ~,~ f_0 \in \dual{\A}, a \in \A, \epsilon > 0\]
erzeugte Topologie auf $\A$ heißt dann \emph{\ssTop}.
\end{defn}

\begin{kor}\label{bem:Einsetz-stetig}
Die \ssTop{} ist gerade so definiert, dass die \emph{Einsetzungsabbildungen} $\eval_a: \dual{\A} \to \CC: f \mapsto f(a)$ für alle $a\in \A$ stetig sind.
\end{kor}

\begin{proof}
Zu $f_0 \in \dual{\A}$, $\epsilon > 0$ betrachte die offene Menge $U(f_0, a, \epsilon) \subseteq \dual{\A}$.
\end{proof}

\begin{bem}
Da $\SpecC(\CC^n)$ aus \Cref{sec:BeispielCn} eine Teilmenge von $\dual{\CC^n}$ ist, können wir auch diesen Raum mit der \ssTop{} versehen. Es ist leicht zu sehen, dass sie in diesem Spezialfall der diskreten Topologie entspricht (betrachte die Umgebungen $U(f_k, e_k, {}^1/_2)$.
\end{bem}

\begin{lemma}\label{lemma:MA}
Sei $\A$ eine \CAlg. Dann ist
\[\SpecC(\A) := \{f: \A \to \CC ~|~ f \text{ stetiger Algebrenhomomorphismus}\} \subset \dual{\A}\]
mit der \ssTop{} ein kompakter Hausdorffraum.
\end{lemma}

\begin{proof}
Die Hausdorffeigenschaft für $\SpecC(\A)$ folgt bereits direkt aus der Definition der \ssTop{} (vgl. \cite[Lemma 2.1.27 b)]{Baer2003}): Sind nämlich $f, g \in \SpecC(\A)$ mit $f \neq g$, d.h. $\exists a\in A: f(a) \neq g(a)$, so setzen wir $\epsilon := \frac{1}{2} |f(a) - g(a)| > 0$. Dann sind $U(f, a, \epsilon), U(g, a, \epsilon)$ disjunkte offene Umgebungen von $f$ bzw. $g$. Also ist $\SpecC(\A)$ hausdorffsch.

Für den Beweis der Kompaktheit benötigen wir dagegen einige Hilfsaussagen:

\begin{figure}[h]
	\tikzstyle{Def} = [rectangle, draw, fill=gray!50, 
    text width=4.5em, text badly centered]
\tikzstyle{Prop} = [rectangle, draw, 
    text centered, rounded corners]
\tikzstyle{Absch} = [rectangle, draw, 
    text centered]
\tikzstyle{keinBeweis} = [rectangle, fill=gray!35, 
    text centered]    
\tikzstyle{Text} = [ 
    text centered]
\tikzstyle{BewTeil} = []

\tikzstyle{line} = [draw, -latex']
\tikzstyle{line2} = [draw]
  
\begin{footnotesize}

\begin{tikzpicture}[node distance = 2cm, auto]
    % Place nodes
	\node [Text] (satz) {$\SpecC(\A)$ ist kompakt};
	\node [below of=satz, node distance=1cm] (hilf-satz) {};
    
    \node [Text, left of=hilf-satz] (Spec0-komp) {$\SpecC(\A)\cup\{0\}$ komp.};    
    \node [keinBeweis, below of=Spec0-komp, text width=10em, node distance=1.5cm] (B0-komp) 
    	{\cref{satz:BA} \\ $\dual{B_1}(0) \subseteq \dual\A$ komp.}; 
    \node [Prop, left of=B0-komp, text width=10em, node distance=4cm] (Spec-in-B0) {\cref{prop:Spec-in-B0} \\ $\SpecC(\A)\cup\{0\} \subseteq \dual{B_1}(0)$};
    \node [Prop, right of=B0-komp, text width=10em, node distance=4cm] (Spec0-abg) {\cref{prop:Spec0-abg} \\ $\SpecC(\A) \cup \{0\} \subseteq \dual{\A}$ abg.};
         
	\node [Prop, right of=hilf-satz, text width=13em, node distance=5cm] (Spec-abg) {\cref{prop:Spec-abg} \\ $\SpecC(\A) \subseteq \SpecC(\A)\cup\{0\}$ abg.};   
	
	\node [Text, below of=Spec0-abg, text width=13em, node distance=1.5cm] (Einsetz-stetig) {\cref{bem:Einsetz-stetig} \\ $\eval_a: \dual{A} \to \CC: \varphi \mapsto \varphi(a)$ stetig};
                                
    % Draw edges
	\path [line] (Spec0-komp) -- (satz);
	\path [line] (B0-komp) -- (Spec0-komp);
	\path [line] (Spec-in-B0) -- (Spec0-komp);
	\path [line] (Spec0-abg) -- (Spec0-komp);
	
	\path [line] (Spec-abg) -- (satz);	
	
	\path [line] (Einsetz-stetig) -| (Spec-abg);      
	\path [line] (Einsetz-stetig) -- (Spec0-abg);      
		
\end{tikzpicture}



\end{footnotesize}
	\caption{Der Aufbau des Beweises zu \cref*{lemma:MA}}
\end{figure}


\begin{prop}
Sei $\A$ eine \CAlg{} und $\dual{\A}$ der Dualraum dazu, versehen mit der schwach-*-Topologie. Sei $0$ die Nullabbildung in $\dual{\A}$ und $\dual{B_1}(0) \subseteq \dual{\A}$ die abgeschlossene Einheitskugel bezüglich der induzierten Norm. Dann gilt:
\begin{propenum}
	\item $\SpecC(\A) \cup \{0\} \subseteq \dual{\A}$ ist abgeschlossen. 		\label{prop:Spec0-abg} 
	\item $\SpecC(\A) \subseteq \SpecC(\A) \cup \{0\}$ ist abgeschlossen.		\label{prop:Spec-abg} 
	\item $\SpecC(\A) \subseteq \dual{B_1}(0)$							\label{prop:Spec-in-B0} 
\end{propenum}
\end{prop}

\begin{proof}Der Beweis zu \ref*{prop:Spec0-abg} und \ref*{prop:Spec-abg} folgt dem zu \cite[Lemma 2.1.27 a), d)]{Baer2003}.
\begin{proofenum} 
	\item %Folgt 2.1.27a
	Für beliebige $a, b \in \A$ definieren wir die folgende Abbildung:
	\[\Phi_{a,b} := \eval_{ab}-\eval_a \eval_b : \dual{\A} \to \CC: f \mapsto f(ab)-f(a)f(b)\]
	Für $f \in \SpecC(\A) \cup \{0\}$ gilt $f(ab) = f(a)f(b)$, die Abbildung $\Phi_{a,b}$ ist also gleich 0 auf $\SpecC(\A) \cup \{0\}$. Da $\Phi_{a,b}$ nach \Cref{bem:Einsetz-stetig} stetig ist, muss $\Phi_{a,b}$ auch auf dem Abschluss von $\SpecC(\A) \cup \{0\}$ identisch Null sein. 
	
Sei nun $f \in \dual{\A} \backslash \{0\}$ eine Abbildung, sodass für alle $a, b \in \A$ gilt: $0 = \Phi_{a,b}(f) = f(ab) - f(a)f(b)$. Damit ist $f$ insbesondere multiplikativ. Außerdem ist $f(e) \neq 0$, denn sonst würde für alle $a \in \A$ gelten $f(a) = f(ea) = f(e)f(a) = 0f(a) = 0$, es wäre also $f = 0$. Daraus wiederum folgt nun, dass $f$ einserhaltend ist, denn es gilt:
	\[f(e) = f(ee) = f(e)f(e) \overset{f(e) \in \CC}{\Longrightarrow} 1 = f(e).\]
Somit ist $f \in \SpecC(\A)$ und es gilt:
	\[\overline{\SpecC(\A) \cup \{0\}} \subseteq \big\{f \in \dual{\A} ~|~ \forall a,b \in \A: \Phi_{a,b}(f) = 0 \big\} \subseteq \SpecC(\A) \cup \{0\} \]
	Also $\overline{\SpecC(\A) \cup \{0\}} = \SpecC(\A) \cup \{0\}$
	
	\item %Folgt 2.1.27d
	Die Abbildung 
	\[\eval_e: \SpecC(\A) \cup \{0\} \to \CC: f \mapsto f(e) = \begin{cases} 1 &, f \in \SpecC(\A) \\ 0 &, f \equiv 0 \end{cases}\]
	ist nach \Cref{bem:Einsetz-stetig} stetig. Also ist $\SpecC(\A) = \eval_1^{-1}(\{1\})$ als Urbild einer abgeschlossenen Menge selbst abgeschlossen.
	
	\item Dieser Beweis folgt dem zu \cite[Lemma IX.2.2]{Werner2011}
	
	\Ann Es gibt ein $f \in \SpecC(\A)$ mit $\norm{f} := \sup\{|f(a)| ~|~ a\in\A, \norm{a} = 1\} > 1$. 
	
	Dann gibt es also ein $a \in \A$ mit $\norm{a} = 1$ und $|f(a)| > 1$, d.h. für $b := \frac{a}{f(a)}$ gilt
	\[ \norm{b} = \norm{\frac{a}{f(a)}} = \frac{\norm{a}}{|f(a)|} < 1 \]
	und
	\[ f(b) = f\left(\frac{a}{f(a)}\right) = \frac{f(a)}{f(a)} = 1.\]
	Setzen wir nun $c := \sum_{k=1}^\infty b^n$ (konvergent, da $||b|| < 1$ und $\A$ vollständig), dann gilt $c = \sum_{k=1}^\infty b^n = b + b\sum_{k=1}^\infty b^n = b+bc$ und damit folgt der Widerspruch:
	\[f(c) = f(b+bc) = f(b) + f(b)f(c) = 1 + f(c)\]
	Also gilt für alle $f \in \SpecC(\A): \norm{f} \leq 1$ und somit ist:
	\[\SpecC(\A) \subseteq \dual{B_1}(0)\]\end{proofenum}
\end{proof}

Mit Hilfe dieser Proposition können wir nun die Kompaktheit von $\SpecC(\A)$ zeigen:

Nach dem Satz von Banach-Alaoglu (\Cref{satz:BA}) ist $\dual{B_1}(0) \subseteq \dual\A$ kompakt. Damit ist $\SpecC(\A)\cup\{0\}$ als abgeschlossene Teilmenge von $\dual{B_1}(0)$ kompakt und ebenso $\SpecC(\A)$ als abgeschlossene Teilmenge von $\SpecC(\A)\cup\{0\}$.
\end{proof}



\subsection{$\stetig(X)$ als \CAlg}\label{sec:CX}

Nachdem wir zu jeder \CAlg{} einen kompakten Hausdorffraum finden können, werden wir als nächstes aus einem solchen wieder eine \CAlg{} konstruieren.

\begin{lemma}\label{lemma:CX}
Sei $X$ ein kompakter Hausdorffraum, dann ist
\[\stetig(X) := \{\tau:X \to \CC ~|~ \tau \text{ stetig}\}\]
mit (für $\tau,\psi \in \stetig(X), x \in X, \lambda \in \CC$):
\[(\tau+\psi)(x) := \tau(x)+\psi(x),\qquad (\lambda\tau)(x) := \lambda \tau(x),\qquad (\tau\cdot \psi)(x) := \tau(x)\psi(x)\]
\[e := (X \to \CC: x \mapsto 1) ,\qquad \norm{\tau} := \underset{x \in X}{\sup}|\tau(x)| ,\qquad (\tau^*)(x) := \overline{\tau(x)}\]
eine \CAlg.
\end{lemma}

\begin{proof}
Da $X$ kompakt ist, sind alle stetigen Funktionen auf $X$ beschränkt und die Norm $\norm{\tau} := \underset{x \in X}{\sup}|\tau(x)|$ damit wohldefiniert. Ferner ist $\stetig(X)$ bezüglich dieser Norm vollständig:

Ist nämlich $(\tau_l)_{l \in \NN}$ eine Chauchy-Folge in $\stetig(X)$, so ist insbesondere für jedes $x \in X$ auch $(\tau_l(x))_{l \in \NN}$ eine Chauchy-Folge in $\CC$. Also können wir die folgende Funktion definieren:
	\[\tau: X \to \CC: x \mapsto \lim_{l\to\infty}\tau_l(x)\]
Sei nun $\epsilon > 0$ und $x \in X$. Da die $\tau_l$ eine Chauchy-Folge bilden, existiert dann ein $L \in \NN$, sodass für alle $l \geq L$ gilt
	\[\sup_{y \in X}|\tau_l(y) - \tau_L(y)| = \norm{\tau_l - \tau_L} < {}^\epsilon/_3\]
und somit auch
	\[\norm{\tau - \tau_L} = \sup_{y \in X}|\lim_{l\to\infty}\tau_l(y) - \tau_L(y)| \leq {}^\epsilon/_3.\]
Das heißt, die $\tau_l$ konvergieren gegen $\tau$. Außerdem ist $\tau_L$ nach Voraussetzung stetig und daher existiert eine offene Umgebung $U \subseteq X$ von $x$, sodass für alle $y \in U$ gilt:
	\[|\tau_L(y) - \tau_L(x)| < {}^\epsilon/_3\]
Damit ist für alle $y \in U$
	\[|\tau(y) - \tau(x)| \leq |\tau(y) - \tau_L(y)| + |\tau_L(y) - \tau_L(x)| + |\tau_L(x) - \tau(x)| < {}^\epsilon/_3 + {}^\epsilon/_3 + {}^\epsilon/_3 = \epsilon\]
und folglich $\tau \in \stetig(X)$.

Schließlich lässt sich leicht nachrechnen, dass durch die oben angegebenen Operatoren ein normierter $\CC$-Vektorraum eine Banachalgebra und eine damit verträgliche Involution definiert werden.
\end{proof}

\subsection{Der Beweis zum Satz von Gelfand-Neumark}

Nach diesen Vorarbeiten sind wir nun in der Lage den Satz von Gelfand-Neumark für kommutative \CAlgn{} mit Eins formal aufschreiben zu können: 

\begin{satz}\label{satz:GN}
Sei $\A$ eine \CAlg, dann ist die Abbildung
\[\AlgIso: \A \to \stetig(\SpecC(\A)): a \mapsto \left(\tau_a: f \mapsto f\left(a\right)\right)\]
ein isometrischer Isomorphismus von C*-Algebren.
\end{satz}

Der Beweis dieses Satzes erfolgt wieder aufgeteilt in mehreren Teilschritten und unter Zuhilfenahme einiger der bereits gezeigten Aussagen. Die Beweise der einzelnen Aussagen folgen dabei - soweit nicht anders angegeben - denen in \cite[S. 49-69]{Baer2003}.
\begin{figure}[h]
\begin{tiny}
	{\crefname{prop}{Prop.}{Prop.}
\crefname{section}{Absch.}{Absch.}
\crefname{kor}{Kor.}{Kor.}
\tikzstyle{Def} = [rectangle, draw, fill=gray!50, 
    text width=4.5em, text badly centered]
\tikzstyle{Prop} = [rectangle, draw, 
    text centered, rounded corners]
\tikzstyle{Absch} = [rectangle, draw, 
    text centered]
\tikzstyle{keinBeweis} = [rectangle, fill=gray!35, 
    text centered]    
\tikzstyle{Text} = [ 
    text centered]
\tikzstyle{BewTeil} = []

\tikzstyle{line} = [draw, -latex']
\tikzstyle{line2} = [draw]

\begin{tikzpicture}[node distance = 6em, auto]

\node [Text] (satz) {\large $\AlgIso: \A \to \stetig(\SpecC(\A)): a \mapsto \left(\tau_a:f \mapsto f(a)\right)$};
	\node [BewTeil, left of=satz, node distance=40em] (legende) {\large Legende:};
		\node [keinBeweis, below of=legende, node distance=2em, text width=1em, text height=1em] (grau) {};
		\node [BewTeil, right of=grau, node distance=10em, text width=17em] (grau-erk) {= ohne Beweis verwendete Aussage};
		\node [Absch, below of=grau, node distance=2em, text width=1em, text height=1em] (viereck) {};
		\node [BewTeil, right of=viereck, node distance=10em, text width=17em] (viereck-erk) {= wesentlicher Beweisschritt};
		\node [Prop, below of=viereck, node distance=2em, text width=1em, text height=1em] (rund) {};
		\node [BewTeil, right of=rund, node distance=10em, text width=17em] (rund-erk) {= Hilfsaussage};
	\node [below of=satz, node distance=3em] (hilf-satz) {};
    
    %Bijektiv-Block
    \node [Absch, left of=hilf-satz, node distance=16em, text width=8em] (bi) {\Cref{sec:Bijektiv} \\ $\AlgIso$ bijektiv};    
    	\node [below of=bi](hilf-bi) {}; 
    	\node [Text, left of=hilf-bi, node distance=8em] (inj) {$\AlgIso$ injektiv};
    			\node [left of=inj, node distance=18em] (links-inj) {};
       		%Isomometrisch
    		\node [Absch, below of=inj, text width=8em] (iso) {\Cref{sec:isometrisch} \\ $\AlgIso$ isometrisch};
				\node [below of=iso](hilf-iso) {};     
    			\node [Prop, left of=hilf-iso, node distance=13em, text width=8em] (RAnorm) 
    			{\Cref{prop:R-gleich-Norm} \\ $R_\A(a) = \norm{a}$};
    					\node [below of=RAnorm, node distance=3em, text width=4em] (unter-RAnorm) {};
    				\node [Text, below of=RAnorm] (hilf-RAnorm) {};    				
    				\node [Prop, below of=hilf-RAnorm, text width=8em] (RAlima) 
    				{\Cref{prop:R-groesser-lima} \\ in Banachalg.: \\ $R\!_\A(a)\!\!\geq\!\!\lim\!\norm{\!a^k\!}^\frac{1}{k}$}; 
    				\node [Prop, below of=RAlima, text width=8em] (konv)
    				{\Cref{prop:Konvergenz} \\ $\alpha_n \to \epsilon < 1$ \\ $\Rightarrow (\alpha_n)^n \to 0$};
       			\node [Prop, right of=hilf-iso, node distance=10em, text width=15em] (fasigma) 
       			{\Cref{prop:Spektrum-von-a} \\ $\{f(a)|f\in\SpecC(\A)\}=\sigma_\A(a)$};
					\node [BewTeil, below of=fasigma] (bew-fasigma) 
					{\qquad für $\lambda \in \sigma_\A(a)$:};
					\node [BewTeil, below of=bew-fasigma, text width=10em, node distance=1.5em] (bew-fasigma1) 
					{\ref{proof:Spektrum-von-a:maxIdeal} $(\lambda e-a)\A \subseteq I_\lambda$ \\ \qquad\!\!\!\!\! max. Ideal};
						\node [keinBeweis, left of=bew-fasigma1, node distance=12em] (zorn) 
						{Lemma v. Zorn (\ref{satz:LZ})};						
					\node [BewTeil, below of=bew-fasigma1, text width=10em, node distance=2.5em] (bew-fasigma2) 
					{\ref{proof:Spektrum-von-a:I-abg} $I_\lambda$ abg.};										    		
							\node [left of=bew-fasigma2, node distance=6.5em] (links-bew-fasigma2) {};
					\node [BewTeil, below of=bew-fasigma2, text width=10em, node distance=2em] (bew-fasigma3) 
					{\ref{proof:Spektrum-von-a:AI-BA} $^\A/_{I_\lambda}$ BA \& Körper};	
						\node [keinBeweis, left of=bew-fasigma3, node distance=12em] (quotient) 
						{Quatientenalg. (\ref{satz:Quotient})};						
					\node [BewTeil, below of=bew-fasigma3, text width=10em, node distance=2em] (bew-fasigma4) 
					{\ref{proof:Spektrum-von-a:Projektion} $f:\A \to {}^\A/_{I_\lambda} \cong \CC$};
						\node [Prop, left of=bew-fasigma4, node distance=12em] (nichtleer) 
						{\Cref{kor:spektrum-nicht-leer}: $\sigma_\A(a) \neq \emptyset$};						
				    
    	\node [Text, right of=hilf-bi, node distance=8em] (sur) {$\AlgIso$ surjektiv};
			\node [below of=sur](hilf-sur) {};     
    		\node [Text, left of=hilf-sur, node distance=5em] (HAabg) {$\AlgIso(\A)$ abg.};
    		\node [Text, right of=hilf-sur, node distance=5em] (HAdicht) {$\AlgIso(\A)$ dicht};
				\node [below of=HAdicht] (hilf-HAdicht) {};
					\node [keinBeweis, right of=hilf-HAdicht, text width=7em, node distance=4.3em] (SW) 
					{Satz v. Stone-Weiherstraß (\ref{satz:SW})};   
				\node [BewTeil, below of=hilf-HAdicht] (bew-HAdicht) {$\SpecC(\A) \dots$};
				\node [BewTeil, below of=bew-HAdicht, text width=8.4em, node distance=2em] (bew-HAdicht1) 
				{$\bullet$ ist kompakt};
						\node [right of=bew-HAdicht1, node distance=7em] (rechts-bew-HAdicht1) {};	
				\node [BewTeil, below of=bew-HAdicht1, text width=8.4em, node distance=2em] (bew-HAdicht2) 
				{$\bullet$ enth. konst. Fkt.};
						\node [right of=bew-HAdicht2, node distance=5em] (rechts-bew-HAdicht2) {};
				\node [BewTeil, below of=bew-HAdicht2, text width=8.4em, node distance=2em] (bew-HAdicht3) 
				{$\bullet$ trennt Pkte.};	
				\node [BewTeil, below of=bew-HAdicht3, text width=8.4em, node distance=2em] (bew-HAdicht4) 
				{$\bullet$ ist selbstadj.};	

	% Sternhom-Block
    \node [Absch, right of=hilf-satz, node distance=11em, text width=10em] (sternhom) 
    {\Cref{sec:CAlgHom} \\ $\AlgIso$ \CAlgHom}; 
    		\node [right of=sternhom] (rechts-sternhom) {}; 
    	\node [Text, below of=sternhom] (stern) {$\AlgIso(a^*) = (\AlgIso(a))^*$};    
    			\node [left of=stern, node distance=5.5em] (links-stern) {}; 
      	\node [Prop, below of=stern, text width=8em] (sternkonj) 
    	{\Cref{prop:Stern-zu-Konjugation} \\ für $f: \A \to \CC$ \\ Alghom.: \\ $f(a^*) = \overline{f(a)}$};
    	\node [Prop, below of=sternkonj, text width=8em] (selbstadj) 
    	{\Cref{prop:Spektrum-reell} \\ für $a$ selbstadj.: \\ $\sigma_\A(a) \subseteq \RR$};
    			\node [below of=selbstadj, node distance=3em, text width=4em] (unter-selbstadj) {};    	
    	\node [Prop, below of=selbstadj, text width=8em] (eEig) 
    	{\Cref{lemma:CAlg-Eigenschaften} \\ $\norm{e} = 1$, $e^* = e$};
    	%Lemma 1.15 (Spec komp.)
    	\node [Absch, below of=eEig, text width=8em] (Speckomp) {\Cref{lemma:MA} \\ $\SpecC(\A)$ komp. HD-Raum}; 
    	\node [keinBeweis, below of=Speckomp, text width=8em] (alaoglu) {Satz v. Banach-Alaoglu (\ref{satz:BA})};     	
    	
    %Alghom	
	\node [Absch, below of=bew-HAdicht4, text width=15em, node distance=6.6em] (Alghom) 
	{\Cref{sec:Algebrenhomomorphismus} \\ $\AlgIso$ Algebrenhomomorphismus};    	

	%stetig
	\node [Absch, below of=nichtleer, text width=8em, node distance=4em] (stetig) 
	{\Cref{lemma:BAlg-Eigenschaften} \\ $+$, $\cdot$, $^{-1}$ stetig}; 


%Pfade:
	%Zu bijektiv:
	\path [line] (inj) -- (bi);
		\path [line] (iso) -- (inj);
			\path [line] (RAnorm) -- (iso);
				\path [line] (RAlima) -- (RAnorm);
					\path [line] (konv) -- (RAlima);
			\path [line] (fasigma) -- (iso);
				\path [line] (bew-fasigma) -- (fasigma);
					\path [line] (zorn) -- (bew-fasigma1);
					\path [line] (quotient) -- (bew-fasigma3);
					\path [line] (nichtleer) -- (bew-fasigma4);
						\path [line] (RAlima.east) -- (nichtleer.west);
	\path [line] (sur) -- (bi);
		\path [line] (HAabg) -- (sur);
			\path [line] (iso) -- (HAabg);
		\path [line] (HAdicht) -- (sur);
			\path [line] (bew-HAdicht) -- (HAdicht);
				\path [line2] (Speckomp) -| (rechts-bew-HAdicht1.center);
					\path [line] (rechts-bew-HAdicht1.center) -- (bew-HAdicht1);
					\path [line] (alaoglu) -- (Speckomp);
	%Zu Sternhom:
	\path [line] (stern) -- (sternhom);
	\path [line] (sternkonj) -- (stern);
		\path [line] (fasigma) -- (sternkonj.west);
	\path [line] (selbstadj) -- (sternkonj);
	\path [line] (eEig) -- (selbstadj);	
	
    %Von Algebrenhomomorphismus:
    \path [line2] (Alghom.east) -| (rechts-sternhom.east);
        \path [line] (rechts-sternhom.east) -- (sternhom);
    \path [line2] (Alghom.north -| rechts-bew-HAdicht2.east) -- (rechts-bew-HAdicht2.east);
    	\path [line] (rechts-bew-HAdicht2.east) -- (bew-HAdicht2);
    \path [line2] (Alghom.west) -| (links-inj.west);
	    \path [line] (links-inj.west) -- (inj);
	\path [line2] (Alghom.north -| hilf-sur) -- (fasigma.south -| hilf-sur);
		\path [line] (fasigma.north -| hilf-sur) |- (HAabg);
	
	%Von stetig:
	\path [line] (stetig.west) -- (RAlima);
	\path [line2] (stetig.east) -| (links-bew-fasigma2.center);
		\path [line] (links-bew-fasigma2.center) -- (bew-fasigma2);
	    
	%Von stern zu selbstadj    
    \path [line2] (stern.west) -- (links-stern.east);
	    \path [line] (links-stern.east) |- (bew-HAdicht4);
	    
	%Von SW:
	\path [line2] (SW) -- (hilf-HAdicht.center);	
	
	%RAlima nach selbstadj
	\path [line2] (RAnorm.south -| unter-RAnorm.east) -- (unter-RAnorm.east);
		\path [line2] (unter-RAnorm) -- (unter-selbstadj);
		\path [line] (unter-selbstadj.west) -- (selbstadj.south -| unter-selbstadj.west);    
	          
\end{tikzpicture}
}
\end{tiny}
	\caption{Der Aufbau des Beweises zum Satz von Gelfand-Neumark (große Version in \cref{sec:DiagramGN})}
\end{figure}

\subsubsection{$\AlgIso$ ist ein Algebrenhomomorphismus}\label{sec:Algebrenhomomorphismus}

\begin{proof}[\Bew{$\AlgIso$ Algebrenhomomorphismus}]Als erstes gilt es zu zeigen, dass die in \Cref{satz:GN} definierte Abbildung $\AlgIso: \A \to \stetig(\SpecC(\A)): a \mapsto \left(\tau_a: f \mapsto f\left(a\right)\right)$ überhaupt wohldefiniert ist, d.h. dass die $\tau_a$ tatsächlich in $\stetig(\SpecC(\A))$ liegen.

Sei also $a \in \A$. Dann gilt für alle $f \in \SpecC(\A)$:
\begin{itemize}
	\item $f: \A \to \CC \Rightarrow \tau_a(f) = f(a) \in \CC$, also $\tau_a: \SpecC(\A) \to \CC$
	\item Für $\epsilon > 0$ ist $U(f, a, \epsilon)$ eine offene Umgebung von $f$ und es gilt:
		\[\forall g \in U(f, a, \epsilon): |\tau_a(f) - \tau_a(g)| = |f(a) - g(a)| < \epsilon\]
\end{itemize}
Also ist wie gewünscht $\tau_a \in \stetig(\SpecC(\A))$.

Leicht zu sehen ist ferner, dass $\AlgIso$ ein tatsächlich Algebrenhomomorphismus ist, denn für alle $a, b \in \A$, $\lambda \in \CC$ und $f \in \SpecC(\A)$ gilt:
	\[\AlgIso(\lambda a + b)(f) = \tau_{\lambda a + b}(f) = f(\lambda a + b) = \lambda f(a) + f(b) = \lambda \tau_a(f) + \tau_b(f) = \left(\lambda \AlgIso(a)+\AlgIso(b)\right)(f),\]
	\[\AlgIso(ab)(f) = \tau_{ab}(f) = f(ab) = f(a)f(b) = \tau_a(f)\tau_b(f) = (\tau_a\tau_b)(f) = \left(\AlgIso(a)\AlgIso(b)\right)(f)\]
und
	\[\AlgIso(e)(f) = \tau_e(f) = f(e) = 1.\]
Also ist $\AlgIso(\lambda a + b) = \lambda \AlgIso(a) + \AlgIso(b)$, $\AlgIso(ab) = \AlgIso(a)\AlgIso(b)$ und $\AlgIso(e) = \left(f \mapsto 1\right)$. Damit ist gezeigt, dass die Abbildung $\AlgIso$ ein Algebrenhomomorphismus ist.

\let\qed\relax
\end{proof}

\subsubsection{$\AlgIso$ ist isometrisch}\label{sec:isometrisch}

Als nächsten Schritt werden wir zeigen, dass $\AlgIso$ isometrisch ist, d.h dass für alle $a \in \A$ gilt:
	\[\norm{a} = \norm{\AlgIso(a)} := \underset{f \in \SpecC(\A)}{\sup}\left|f(a)\right|\]
Dazu werden wir die beiden Hilfsaussagen $\norm{a} = R_\A(a) := \sup\{|\lambda| ~|~ \lambda \in \sigma_\A(a)\}$ und $\sigma_\A(a) = \{f(a) ~|~ f \in \SpecC(\A)\}$ verwenden, die wir in den folgenden Propositionen zeigen werden. Einige der Aussagen werden wir später noch einmal benötigen und sie daher allgemeiner für Banachalgebren beweisen.

\begin{defn}[Spektrum] %Folgt Def 2.1.11
Sei $\A$ eine Banachalgebra und $a \in \A$. Dann heißt die Menge
	\[\sigma_\A(a) := \{\lambda \in \CC ~|~ \lambda e - a \notin \A^\times\}\]
\emph{Spektrum von $a$} und die Zahl
	\[R_\A(a) := \sup\{|\lambda| ~|~ \lambda \in \sigma_\A(a)\}\]
\emph{Spektralradius von $a$}. 
\end{defn}


\begin{prop}\label{prop:Konvergenz}
Ist $(\alpha_n)_{n\in\NN}$ eine Folge in $\RR$, die gegen ein $\alpha \in [0,1[$ konvergiert, so konvergiert $(\alpha_n)^n$ gegen $0$
\end{prop}

\begin{proof}
Zunächst gibt es wegen $\alpha_n \xlongrightarrow{n\to\infty} \alpha$ ein $K \in \NN$, sodass für alle $n\geq K$ gilt:
	\[|\alpha_n - \alpha| < \frac{1-\alpha}{2}.\]
Außerdem ist $\frac{1+\alpha}{2} < \frac{1+1}{2} = 1$ und daher $\left(\frac{1+\alpha}{2}\right)^n \xlongrightarrow{n\to\infty} 0$. Also existiert für jedes $\epsilon > 0$ ein $N \in \NN$, sodass
	\[\forall n \geq N: \left(\frac{1+\alpha}{2}\right)^n < \epsilon.\]
Ohne Einschränkung können wir dabei $N \geq K$ wählen. Damit gilt für alle $n \geq N$:
	\[\left|\left(\alpha_n\right)^n\right| = \left|\alpha_n\right|^n \leq \left(\left|\alpha_n - \alpha\right| + \left|\alpha\right|\right)^n \leq \left(\frac{1-\alpha}{2} + \alpha\right)^n = \left(\frac{1-\alpha+2\alpha}{2}\right)^n = \left(\frac{1+\alpha}{2}\right)^n < \epsilon\]
Also konvergiert $(\alpha_n)^n$ gegen $0$.
\end{proof}


\begin{prop}\label{prop:R-groesser-lima}
Sei $\A$ eine Banach-Algebra und $a \in \A$. Dann gilt:
\[ R_\A(a) \geq \lima := \underset{k\to\infty}{\lim}\norm{a^k}^\frac{1}{k}\]
\end{prop}

\begin{proof}Zunächst zum Sonderfall $a = 0$: Es ist $0 e - a = 0 - 0 = 0 \notin \A^\times$, also $0 \in \sigma_\A(a)$. Folglich gilt:
	\[R_\A(a) = \sup\{|\lambda| ~|~ \lambda \in \sigma_\A(a)\} \geq 0 = \underset{k\to\infty}{\lim}\norm{0^k}^\frac{1}{k} = \underset{k\to\infty}{\lim}\norm{a^k}^\frac{1}{k}\]

Im Folgenden können wir uns daher auf $a \neq 0$ beschränken. Hierfür zeigen wir zunächst,
\begin{proofenum}
	\item \label{proof:R-groesser-lima:limaexists}
		$\lim_{k\to\infty}\norm{a^k}^\frac{1}{k}$ existiert und $L(a) \leq \norm{a^n}^\frac{1}{n}$ gilt für alle $n \in \NN$.
\setcounter{temp}{\value{proofenumi}}
\end{proofenum}
Dann nehmen wir an $\lambda e - a$ wäre für alle $\lambda \in \CC$ mit $|\lambda| \geq \lima$ invertierbar und zeigen für alle solche $\lambda$, beliebige $n \in \NN$ und $\omega := \exp{\frac{2\pi i}{n}} \in \CC$ die folgenden Aussagen: 
\begin{proofenum}
\setcounter{proofenumi}{\value{temp}}
	\item \label{proof:R-groesser-lima:invertierbar}
		$\frac{1}{n}\sum_{k=1}^n \left(e-\frac{\omega^k a}{\lambda}\right)^{-1} \in \A$ ist das Inverse von $e - \frac{a^n}{\lambda^n}$.
	\item \label{proof:R-groesser-lima:abschaetzung}
		Es gibt ein von $\lambda$ unabhängiges $r_\textup{konst} \in \RR$, sodass folgende Abschätzung gilt:
		\[\norm{\left(e - \frac{a^n}{\lima^n}\right)^{-1} - \left(e - \frac{a^n}{\lambda^n}\right)^{-1}} \leq r_\textup{konst}\left|\lima - \lambda\right|\]
	\item \label{proof:R-groesser-lima:nach0}
		Es gilt $\left(\frac{\norm{a^n}^\frac{1}{n}}{\lima}\right)^n = \frac{\norm{a^n}}{\lima^n} \xlongrightarrow{n\to\infty} 0$
\end{proofenum}
Dann ist jedoch \ref{proof:R-groesser-lima:nach0} ein Widerspruch zu \ref{proof:R-groesser-lima:limaexists}, die Annahme also falsch und es folgt die zu zeigende Aussage.

Zu \ref{proof:R-groesser-lima:limaexists}: Hierzu zeigen wir, dass die $\norm{a^k}^\frac{1}{k}$ eine monoton fallende Folge bilden. Da diese Folge außerdem durch $0$ nach unten beschränkt ist, muss der Limes damit bereits existieren.
\begin{align*}
	\norm{a^k}^\frac{1}{k} &= \norm{a^k}^\frac{1}{k}\cdot\norm{a}^{-\frac{1}{k+1}+\frac{1}{k+1}} = \norm{a^k}^\frac{1}{k}\cdot\left(\norm{a}^k\right)^{-\frac{1}{k(k+1)}}\cdot\norm{a}^{\frac{1}{k+1}} \geq \\
	&\geq \norm{a^k}^\frac{1}{k}\cdot\norm{a^k}^{-\frac{1}{k(k+1)}}\cdot\norm{a}^{\frac{1}{k+1}} = \norm{a^k}^{\frac{k+1-1}{k(k+1)}}\cdot\norm{a}^{\frac{1}{k+1}} = \left(\norm{a^k}\norm{a}\right)^{\frac{1}{k+1}} \geq \\
	&\geq \norm{a^{k+1}}^\frac{1}{k+1}
	\end{align*}
Zusätzlich ergibt sich hieraus für alle $n \in \NN$ die Abschätzung $\lima = \underset{k\to\infty}{\lim}\norm{a^k}^\frac{1}{k} \leq \norm{a^n}^\frac{1}{n}$.

Für die folgenden Schritte nehmen wir nun an, $\lambda e - a$ wäre für alle $|\lambda| \geq \lima$ invertierbar.

zu \ref{proof:R-groesser-lima:invertierbar}:
Seien $|\lambda| \geq \lima$ und $n,k \in \NN$ beliebig, dann ist $\left|\frac{\lambda}{\omega^k}\right| = \frac{|\lambda|}{1} \geq \lima$, also $e-\frac{\omega^k a}{\lambda} = \frac{\omega^k}{\lambda} \left(\frac{\lambda}{\omega^k}e - a\right) \in \A^\times$ nach Annahme. Somit existiert $\frac{1}{n}\sum_{k=1}^n \left(e-\frac{\omega^k a}{\lambda}\right)^{-1} \in \A$.

Wegen $\left(\omega^{k}\right)^n = \exp{\frac{2\pi i kn}{n}} = \exp{2\pi i k} = 1$ ist
\begin{align*}
	e - \frac{a^n}{\lambda^n} &= \left(\frac{\omega^k a}{\lambda}\right)^0 - \left(\frac{\omega^k a}{\lambda}\right)^n = \sum_{l=1}^n \left[\left(\frac{\omega^k a}{\lambda}\right)^{l-1} - \left(\frac{\omega^k a}{\lambda}\right)^l\right] = \\
		&=\left(e - \frac{\omega^k a}{\lambda}\right)\sum_{l=1}^n \left(\frac{\omega^k a}{\lambda}\right)^{l-1}. \quad (\#)
\end{align*}
Ferner gilt $\sum_{k=1}^n \left(\omega^{l-1}\right)^k = \begin{cases}1 &,l=1 \\ 0 &,l\in\{2, \dots, n\}\end{cases}$ ($\sim$), denn (vgl. \cite[S. 10-8,10-9]{Einheitswurzeln}):

Ist $l\in\{2, \dots, n\}$, so ist $\omega^{l-1} = \exp{\frac{2\pi i(l-1)}{n}} \neq 1$ und $\left(\omega^{l-1}\right)^n = 1$. Es folgt
	\[0 = \left(\omega^{l-1}\right)^n - 1 = \left(\omega^{l-1} - 1\right)\left(\left(\omega^{l-1}\right)^{n-1} + \dots + \omega^{l-1} + 1\right)\]
und damit
	\[0 = \left(1 + \left(\omega^{l-1}\right)^{n-1} + \dots + \omega^{l-1}\right) = \left(\left(\omega^{l-1}\right)^{n} + \left(\omega^{l-1}\right)^{n-1} + \dots + \omega^{l-1}\right) = \sum_{k=1}^n \left(\omega^{l-1}\right)^k.\]

Zusammen ergibt das die Behauptung aus \ref{proof:R-groesser-lima:invertierbar}:
\begin{align*}
	\left(\frac{1}{n}\sum_{k=1}^n \left(e-\frac{\omega^k a}{\lambda}\right)^{-1}\right) \left(e - \frac{a^n}{\lambda^n}\right) \overset{(\#)}{=} \left(\frac{1}{n}\sum_{k=1}^n \left(e-\frac{\omega^k a}{\lambda}\right)^{-1}\right) \left(e - \frac{\omega^k a}{\lambda}\right)\sum_{l=1}^n \left(\frac{\omega^k a}{\lambda}\right)^{l-1} = \\
	= \frac{1}{n}\sum_{k=1}^n \sum_{l=1}^n \left(\frac{\omega^k a}{\lambda}\right)^{l-1} = \frac{1}{n} \sum_{l=1}^n \left(\frac{a}{\lambda}\right)^{l-1} \sum_{k=1}^n \left(\omega^{l-1}\right)^k \overset{(\sim)}{=} \frac{1}{n} \sum_{l=1}^n \left(\frac{a}{\lambda}\right)^{1-1} = e
\end{align*}

zu \ref{proof:R-groesser-lima:abschaetzung}: Seien wieder $|\lambda| \geq \lima$ und $n,k \in \NN$ beliebig, dann gilt:

\begin{align*}
\left(\frac{\lima}{\omega^k}e-a\right)^{-1} 
	\left(-\lima\right. &+ \left.\lambda\right)\frac{a}{\omega^k} 
	\left(\frac{\lambda}{\omega^k}e - a\right)^{-1} = \\
= \left(e-\frac{\omega^k}{\lima}a\right)^{-1} 
	\frac{\omega^k}{\lima} \left(-\lima\right. &+ \left.\lambda\right)\frac{a}{\omega^k} \frac{\omega^k}{\lambda}
	\left(e - \frac{\omega^k}{\lambda}a\right)^{-1} = \\
= \left(e-\frac{\omega^k}{\lima}a\right)^{-1} 
	\left(-\frac{\omega^k}{\lambda}a \right.&+\left. \frac{\omega^k}{\lima}a\right)
	\left(e - \frac{\omega^k}{\lambda}a\right)^{-1} = \\
= \left(e-\frac{\omega^k}{\lima}a\right)^{-1} 
	\left(\left(e -\frac{\omega^k}{\lambda}a\right) \right.&-\left. \left(e - \frac{\omega^k}{\lima}a\right)\right)
	\left(e - \frac{\omega^k}{\lambda}a\right)^{-1} = \\	
= \left(e-\frac{\omega^k}{\lima}a\right)^{-1} 
	\Bigg(e &- \left(e - \frac{\omega^k}{\lima}a\right)\left(e - \frac{\omega^k}{\lambda}a\right)^{-1}\Bigg)
	 = \\
= \left(e-\frac{\omega^k}{\lima}a\right)^{-1} &- \left(e - \frac{\omega^k}{\lambda}a\right)^{-1} \quad (\star)
\end{align*}

Daraus folgt mit Hilfe von $\left(e - \frac{a^n}{\lambda^n}\right)^{-1} = \frac{1}{n}\sum_{k=1}^n \left(e-\frac{\omega^k a}{\lambda}\right)^{-1}$ aus \ref{proof:R-groesser-lima:invertierbar}:
\begin{align*}
\norm{\left(e-\frac{a^n}{\lima^n}\right)^{-1} - \left(e - \frac{a^n}{\lambda^n}\right)^{-1}} 
	\overset{\ref{proof:R-groesser-lima:invertierbar}}{=} \norm{\frac{1}{n}\sum_{k=1}^n\left(e-\frac{\omega^k}{\lima}a\right)^{-1} - \frac{1}{n}\sum_{k=1}^n\left(e - \frac{\omega^k}{\lambda}a\right)^{-1}} \leq \\
	\leq \frac{1}{n}\sum_{k=1}^n \norm{\left(e-\frac{\omega^k}{\lima}a\right)^{-1} - \left(e - \frac{\omega^k}{\lambda}a\right)^{-1}} = \\
	\overset{(\star)}{=} \frac{1}{n}\sum_{k=1}^n \norm{\left(-\lima + \lambda\right)\frac{a}{\omega^k} \left(\frac{\lima}{\omega^k}e-a\right)^{-1} \left(\frac{\lambda}{\omega^k}e - a\right)^{-1}} \leq \\
	\leq \frac{1}{n}\sum_{k=1}^n \left|-\lima + \lambda\right| \cdot \frac{\norm{a}}{\left|\omega^k\right|} \cdot \left(\underset{|\lambda| \geq \lima}{\sup}\norm{(\lambda e - a)^{-1}}\right)^2 = \\
	= \left|\lima - \lambda\right| \cdot \norm{a} \cdot \underset{|\lambda| \geq \lima}{\sup}\norm{(\lambda e - a)^{-1}}^2	
\end{align*}

Falls $\sup_{|\lambda| \geq \lima}\norm{(\lambda e - a)^{-1}}$ endlich ist, dann wäre die Aussage \ref{proof:R-groesser-lima:abschaetzung} hiermit gezeigt (setze $r_\textup{konst} := \lima \cdot \sup_{|\lambda| \geq \lima}\norm{(\lambda e - a)^{-1}}^2$).

Für $|\lambda| \geq 2\norm{a}$ ist $\norm{\frac{a}{\lambda}} = \frac{\norm{a}}{|\lambda|} \leq \frac{\norm{a}}{2\norm{a}} = \frac{1}{2}$ und daher $\sum_{k=0}^\infty \left(\frac{a}{\lambda}\right)^k$ absolut konvergent. Außerdem gilt
	\[(\lambda e - a)\cdot \frac{1}{\lambda}\sum_{k=0}^\infty \left(\frac{a}{\lambda}\right)^k = \sum_{k=0}^\infty \left(\frac{a}{\lambda}\right)^k - \sum_{k=0}^\infty \left(\frac{a}{\lambda}\right)^{k+1} = 1\]
und somit die Abschätzung
	\[\norm{(\lambda e - a)^{-1}} = \norm{\frac{1}{\lambda}\sum_{k=0}^\infty \left(\frac{a}{\lambda}\right)^k} \leq \frac{1}{|\lambda|}\sum_{k=0}^\infty\norm{\frac{a}{\lambda}}^k \leq \frac{1}{2\norm{a}}\sum_{k=0}^\infty\left|\frac{1}{2}\right|^k = \frac{1}{\norm{a}}. \]

Also ist $(\lambda e - a)^{-1}$ auf $\{\lambda \in \CC ~|~ |\lambda| \geq 2\norm{a}\}$ durch $\frac{1}{\norm{a}}$ beschränkt.

Da die Abbildung $\lambda \mapsto (\lambda e - a)^{-1}$ stetig ist(als Verknüpfung der stetigen Abbildungen Skalarmultiplikation, Addition und Inversenbildung (siehe \cref{lemma:BAlg-Eigenschaften})), ist sie auch auf dem verbleibenden Kompaktum $\{\lambda \in \CC ~|~ 2\norm{a} \geq |\lambda| \geq \lima\}$ beschränkt. 

Somit ist $\sup_{|\lambda| \geq \lima}\norm{(\lambda e - a)^{-1}}$ tatsächlich endlich.


zu \ref{proof:R-groesser-lima:nach0}:
Aufgrund der Stetigkeit von Inversenbildung und Addition (\cref{lemma:BAlg-Eigenschaften}) gilt für eine Folge $(a_n)_{n\in\NN} \subset \A$ mit $e - a_n \in \A^\times$ für alle $n$:
	\[a_n \xlongrightarrow{n\to\infty} 0 \iff e - a_n  \xlongrightarrow{n\to\infty} e \iff (e-a_n)^{-1} \xlongrightarrow{n\to\infty} e\]

Für beliebiges $l \in \NN$ ist nun 
	\[\underset{n\to \infty}{\lim}\frac{\norm{a^n}^\frac{1}{n}}{\lima+\frac{1}{l}} = \frac{\underset{n\to \infty}{\lim}\norm{a^n}^\frac{1}{n}}{\underset{k\to\infty}{\lim}\norm{a^k}^\frac{1}{k}+\frac{1}{l}} < 1\]
und es folgt mit \cref{prop:Konvergenz}
	\[\norm{\frac{a^n}{\left(\lima+\frac{1}{l}\right)^n}} = \frac{\norm{a^n}}{\left|\lima+\frac{1}{l}\right|^n} = \left(\frac{\norm{a^n}^\frac{1}{n}}{\lima+\frac{1}{l}}\right)^n \xlongrightarrow{n\to\infty} 0.\]
Also gilt auch
	\[\frac{a^n}{\left(\lima+\frac{1}{l}\right)^n} \xlongrightarrow{n\to\infty} 0 \qquad \text{und damit} \qquad \left(e - \frac{a^n}{(\lima+\frac{1}{l})^n}\right)^{-1} \xlongrightarrow{n\to\infty} e.\]

Ferner haben wir aus \ref{proof:R-groesser-lima:abschaetzung} die Abschätzung:
	\[\norm{\left(e - \frac{a^n}{\lima^n}\right)^{-1} - \left(e - \frac{a^n}{(\lima + \frac{1}{l})^n}\right)^{-1}} \leq r_\textup{konst}\left|\lima-\left(\lima+\frac{1}{l}\right)\right| = \frac{r_\textup{konst}}{l}\]
Zusammen ergibt das
	\begin{align*}\frac{r_\textup{konst}}{l} &\geq \underset{n\to\infty}{\lim} \norm{\left(e - \frac{a^n}{\lima^n}\right)^{-1} - \left(e - \frac{a^n}{(\lima + \frac{1}{l})^n}\right)^{-1}} = \\
	&= \underset{n\to\infty}{\lim} \norm{\left(e - \frac{a^n}{\lima^n}\right)^{-1} - e}.
	\end{align*}
Da dies für beliebige $l \in \NN$ gilt, folgt
	\[\underset{n\to\infty}{\lim} \norm{\left(e - \frac{a^n}{\lima^n}\right)^{-1} - e} = 0\]
und hieraus
	\[\left(e - \frac{a^n}{\lima^n}\right)^{-1} \xlongrightarrow{n\to\infty} e, \text{ also } \frac{a^n}{\lima^n} \xlongrightarrow{n\to\infty} 0 \text{ und } \frac{\norm{a^n}}{\lima^n} \xlongrightarrow{n\to\infty} 0\]
was die Behauptung \ref{proof:R-groesser-lima:nach0} ist.	

Daraus jedoch entsteht ein Widerspruch, denn aus \ref{proof:R-groesser-lima:limaexists} ergibt sich: 
	\[\frac{\norm{a^n}}{\lima^n} \geq \frac{\norm{a^n}}{\norm{a^n}} = 1 \text{ für alle } n\in \NN\]
Somit muss die Annahme falsch gewesen sein und es gibt doch ein $\lambda \in \CC$ mit $|\lambda| \geq \lima$ und $\lambda e - a \notin \A^\times$, d.h. $\lambda \in \sigma_\A(a)$. Daher gilt die Aussage der Proposition:
	\[R_\A(a) = \sup\{|\lambda| ~|~ \lambda \in \sigma_\A(a)\} \geq \lima\]
\end{proof}


\begin{kor}\label{kor:spektrum-nicht-leer}
Sei $\A$ eine Banach-Algebra, $a \in \A$. Dann ist $\sigma_\A(a) \neq \emptyset$.
\end{kor}

\begin{proof}
Im Beweis zu \cref{prop:R-groesser-lima} haben wir gezeigt, dass es ein $\lambda \in \sigma_\A(a)$ gibt, welches größer als $\lima$ ist. Damit ist $\sigma_\A(a)$ insbesondere auch nicht-leer.
\end{proof}

\begin{prop}\label{prop:R-gleich-Norm}
Ist $\A$ eine \CAlg{} und $a \in \A$, so gilt: 
	\[R_\A(a) = \norm{a}\]
\end{prop}

\begin{proof}

\begin{itemize}
	\item[\glqq$\geq$\grqq] Aus \cref{prop:R-groesser-lima} wissen wir bereits, dass $R_\A(a) \geq \lima = \underset{k\to\infty}{\lim}\norm{a^k}^\frac{1}{k}$ gilt. Also genügt es zu zeigen, dass $\underset{k\to\infty}{\lim}\norm{a^k}^\frac{1}{k} = \norm{a}$ ist.
	
Zunächst gilt $\norm{a^{2^k}}^\frac{1}{2^k} = \norm{a}$, denn für $k \in \NN$ ist
	\begin{align*}
	\norm{(a^*a)^{2^k}} &= \norm{(a^*a)^{2^{k-1}}(a^*a)^{2^{k-1}}} = \norm{((a^*a)^*)^{2^{k-1}}(a^*a)^{2^{k-1}}} = \\
	&= \norm{\left((a^*a)^{2^{k-1}}\right)^*(a^*a)^{2^{k-1}}} = \norm{(a^*a)^{2^{k-1}}}^2.
	\end{align*}
und damit (unter wiederholter Anwendung dieser Aussage)
	\begin{align*}
	\norm{a^{2^k}}^2 &= \norm{\left(a^{2^k}\right)^*a^{2^k}} = \norm{(a^*a)^{2^k}} = \norm{(a^*a)^{2^{k-1}}}^2 = \dots = \norm{(a^*a)^{2^{0}}}^{2^k} = \\
	&= \norm{a*a}^{2^k} = \left(\norm{a}^2\right)^{2^k} = \norm{a}^{2^{k+1}}. \quad (\#)
	\end{align*}	
	
Nachdem wir im Beweis zu \cref{prop:R-groesser-lima} in \ref{proof:R-groesser-lima:limaexists} bereits gezeigt haben, dass die $\norm{a^k}^\frac{1}{k}$ eine absteigende Folge bilden, ergibt sich hieraus
	\[\norm{a} \overset{(\#)}{=} \norm{a^{2^k}}^\frac{1}{2^k} \overset{\ref*{prop:R-groesser-lima}\ref*{proof:R-groesser-lima:limaexists}}{\leq} \norm{a^{k}}^\frac{1}{k} \overset{\ref*{prop:R-groesser-lima}\ref*{proof:R-groesser-lima:limaexists}}{\leq} \norm{a^1}^1 = \norm{a}\]
und folglich gilt
	\[R_\A(a) \geq \underset{k\to\infty}{\lim}\norm{a^k}^\frac{1}{k} = \underset{k\to\infty}{\lim}\norm{a} = \norm{a}.\]	
	
	
	\item[\glqq$\leq$\grqq] Sei $\lambda \in \CC$ mit $|\lambda| > \norm{a}$. Wir wollen zeigen, dass dann $\lambda e - a$ invertierbar ist.

Zunächst ist
	\[\norm{\frac{a}{\lambda}} = \frac{\norm{a}}{|\lambda|} < 1\]
Also konvergiert die folgende geometrische Reihen absolut
	\[\frac{1}{\lambda}\sum_{m=0}^\infty\left(\frac{a}{\lambda}\right)^m = \sum_{m=0}^\infty\left(\frac{a^m}{\lambda^{m+1}}\right) \]
und ist das Inverse zu $\lambda e - a$, denn:
	\[(\lambda e - a)\cdot \sum_{m=0}^\infty\frac{a^m}{\lambda^{m+1}} = \sum_{m=0}^\infty\left(\frac{a^m}{\lambda^{m}}\right) - \sum_{m=0}^\infty\left(\frac{a^{m+1}}{\lambda^{m+1}}\right) = \frac{a^0}{\lambda^0} +  \sum_{m=1}^\infty\left(\frac{a^m}{\lambda^{m}}\right) - \sum_{m=1}^\infty\left(\frac{a^m}{\lambda^{m}}\right) = e\]
	
Insgesamt ist damit gezeigt: Für alle $\lambda \in \CC$ folgt aus $|\lambda| > \norm{a}$, dass $\lambda e - a$ invertierbar ist und daher $\lambda$ nicht in $\sigma_\A(a)$ liegt. Daraus ergibt sich die Aussage:
	\[R_\A(a) = \sup\{|\lambda| ~|~ \lambda \in \sigma_\A(a)\} \leq \norm{a}\]
\end{itemize}	
\end{proof}


Für die nächste Proposition benötigen wir die folgende Definition:

\begin{defn}[Ideal] 
Sei $\A$ eine Banachalgebra. Dann heißt eine Teilmenge $I \subseteq \A$ ein \emph{Ideal von $\A$}, wenn gilt:
\begin{itemize}
	\item $x,y \in I \Rightarrow x+y \in I$
	\item $x \in I, a \in \A \Rightarrow a \cdot x \in I$
\end{itemize}
Ist zusätzlich
\begin{itemize}
	\item $I \subsetneq \A$,
\end{itemize}
so heißt $I$ \emph{echtes Ideal}. Gilt außerdem noch
\begin{itemize}
	\item $\forall J \subseteq \A \text{ Ideal}: I \subsetneq J \Rightarrow J = \A,$
\end{itemize}
dann ist $I$ ein \emph{maximales Ideal}.
\end{defn}

\begin{bem}
Da Banachalgebren hier immer ein Einselement besitzen, folgt die Abgeschlossenheit eines Ideals unter Skalarmultiplikation bereits aus den beiden genannten Eigenschaften. Ist nämlich $\lambda \in \CC$ und $x \in I \subseteq \A$, so ist $\lambda e \in \A$ und daher $\lambda x = (\lambda e) x \in I$.
\end{bem}

\begin{prop}\label{prop:Spektrum-von-a}
Sei $\A$ eine \CAlg, $a \in \A$. Dann gilt:
\[ \{f(a) ~|~ f \in \SpecC(\A)\} = \sigma_\A(a)\]
\end{prop}

\begin{proof}
\begin{itemize}
\item[\glqq$\subseteq$\grqq]%Folgt Bem 2.1.25
 \Ann $\exists f \in \SpecC(\A): f(a) \notin \sigma_\A(a)$
	
Dann ist $f(a)\cdot e - a$ invertierbar, d.h. es gibt ein $b \in \A$, sodass $(f(a)\cdot e - a)\cdot b =e$. Damit aber folgt der Widerspruch:
	\[1 = f(e) = f((f(a)\cdot e - a)\cdot b) = (f(a)\cdot f(e) - f(a))\cdot f(b) = 0\cdot f(b) = 0\]
	
\item[\glqq$\supseteq$\grqq]%Folgt Lemma 2.1.30
Sei $\lambda \in \sigma_\A(a)$ beliebig, dann suchen wir ein $f \in \SpecC(\A)$ mit $f(a) = \lambda$. Die Konstruktion dieses Elements erfolgt in mehreren Schritten:
\begin{proofenum}
	\item \label{proof:Spektrum-von-a:maxIdeal}
		Es gibt ein maximales Ideal $I_\lambda$ mit $(\lambda e-a)\A \subseteq I_\lambda$.
	\item \label{proof:Spektrum-von-a:I-abg}
		$I_\lambda \subsetneq \A$ ist abgeschlossen.
	\item \label{proof:Spektrum-von-a:AI-BA}
		$^\A/_{I_\lambda}$ ist eine Banachalgebra mit $\left({}^\A/_{I_\lambda}\right)^\times = {}^\A/_{I_\lambda} \backslash \{[0]\}$.
	\item \label{proof:Spektrum-von-a:Projektion}
		Die Projektion $f: \A \to ~^\A/_{I_\lambda} \cong \CC$ ist der gesuchte Homomorphismus.
\end{proofenum}

Zu \ref{proof:Spektrum-von-a:maxIdeal}: 
Wir definieren zunächst die Menge
	\[(\lambda e-a)\A := \{(\lambda\cdot e - a)\cdot b ~|~ b \in \A\} \subset \A. \] 
Diese ist offensichtlich ein Ideal. Ferner ist $\lambda \in \sigma_\A(a)$ und deswegen $\lambda e-a \notin \A^\times$. Damit ist $e \notin (\lambda e-a)\A$ (sonst wäre $e = (\lambda e - a)\cdot b$ für ein $b \in \A$ und damit $\lambda e-a \in \A^\times$). Also ist $(\lambda e-a)\A$ sogar ein echtes Ideal.

Betrachte nun die Menge aller echten Ideale, die $(\lambda e-a)\A$ enthalten:
	\[M := \{I \subsetneq A \text{ Ideal} ~|~ (\lambda e-a)\A \subseteq I\}\]
Zusammen mit der Inklusion $\subseteq$ als Ordnungsrelation ist dies eine partiell geordnete Menge. 

Falls diese Menge ein maximales Element besitzt, so wäre dieses eine passende Wahl für das gesuchte maximale Ideal $I_\lambda$. Dass ein solches maximales Element auch tatsächlich existiert, zeigen wir mit Hilfe des Lemmas von Zorn (\ref{satz:LZ}):

\Beh Sei $N \subseteq M$ eine nicht-leere\footnote{Für den Fall der leeren Teilmenge (die trivialerweise total geordnet ist), kann direkt $(\lambda e-a)\A$ selbst als obere Schranke gewählt werden.} total geordnete Teilmenge. Dann ist $J := \bigcup_{I \in N} I$ eine obere Schranke für $N$.
\begin{itemize}
	\item[$\bullet$] $J$ ist ein Ideal, denn:
	
	Sind $x,y \in J$, dann gibt es $I_x, I_y \in N$, sodass $x \in I_x, y \in I_y$. Da $N$ total geordnet ist bezüglich der Inklusion, gilt außerdem oBdA $I_x \subseteq I_y$. Also ist $x,y \in I_y$ und es folgt (da $I_y$ ein Ideal ist): $x+y \in I_y \subseteq J$.
	
	Außerdem gilt für alle $a \in \A$, dass $a\cdot y \in I_y \subseteq J$.
	
	\item[$\bullet$] $J$ ist ein echtes Ideal, denn:
	
	Angenommen es wäre $J = \A$. Dann wäre insbesondere $e \in J$ und damit $e \in I$ für ein $I \in N$. Da $I$ ein Ideal ist, wäre damit aber $a = a\cdot e \in I$ für jedes $a \in \A$.
	
	Das heißt es wäre $I = \A$ kein echtes Ideal, was im Widerspruch zur Definition von $N$ steht.
	
	\item[$\bullet$] $(\lambda e-a)\A$ ist in $J$ enthalten, denn:
	
	$N$ enthält wenigstens ein $I$ aus $M$. Dieses $I$ enthält nach Definition das Ideal $(\lambda e-a)\A$ und damit enthält auch $J$ dieses Ideal.
\end{itemize}
Also liegt $J$ in $M$ und ist offensichtlich eine obere Schranke für $N$, was die Behauptung zeigt.

Damit sind die Voraussetzung des Lemmas von Zorn erfüllt. Es folgt daher, dass $M$ ein maximales Element enthält, welches also gerade die in \ref{proof:Spektrum-von-a:maxIdeal} gewünschte Eigenschaft hat. Wir nennen es daher $I_\lambda$.

Zu \ref{proof:Spektrum-von-a:I-abg}:
Da Addition und Multiplikation in $\A$ nach \cref{lemma:BAlg-Eigenschaften} stetig ist, ist auch $\overline{I_\lambda}$, der Abschluss von $I_\lambda$, ein Ideal. $\overline{I_\lambda}$ enthält aber mit $I_\lambda$ ein maximales Ideal und es                        muss folglich $\overline{I_\lambda} = I_\lambda$ oder $\overline{I_\lambda} = \A$ gelten.

Wäre $\overline{I_\lambda} = \A$, so läge $I_\lambda$ dicht in $\A$ und folglich wäre $I_\lambda \cap \{b \in \A ~|~ \norm{b-e} < 1\} \neq \emptyset$. Sei also $b$ aus $I_\lambda \cap \{b \in \A ~|~ \norm{b-e} < 1\}$, dann ist $\sum_{k=0}^\infty(e-b)^k$ absolut konvergent und das Inverse zu $b$, denn:
	\[b\sum_{k=0}^\infty(e-b)^k = (b-e)\sum_{k=0}^\infty(e-b)^k + e\sum_{k=0}^\infty(e-b)^k = -\sum_{k=0}^\infty(e-b)^{k+1} +\sum_{k=0}^\infty(e-b)^k = 1\]
Damit ist aber $\A = I_\lambda$, denn für beliebiges $c \in \A$ gilt:
	\[c = ce = cb\sum_{k=0}^\infty(e-b)^k = \left(c\sum_{k=0}^\infty(e-b)^k\right) b \in I_\lambda\]
Das jedoch steht im Widerspruch zu \ref{proof:Spektrum-von-a:maxIdeal} ($I_\lambda$ ist ein echtes Ideal). Daher muss $\overline{I_\lambda} = I_\lambda$ gelten und $I_\lambda$ ist tatsächlich abgeschlossen.


Zu \ref{proof:Spektrum-von-a:AI-BA}:
Da $I_\lambda$ also abgeschlossen ist, ist $^\A/_{I_\lambda}$ mit der Norm $\norm{[b]} := \underset{x \in I_\lambda}{\inf}\norm{b+x}$ nach \cref{satz:Quotient} ein vollständiger normierter Vektorraum. Außerdem erhalten wir durch $[b]\cdot [c] := [bc]$ eine Multiplikation (mit Einselement $[e]$). Da $I_\lambda$ als Ideal auch unter Multiplikation abgeschlossen ist, ist diese Operation auf $^\A/_{I_\lambda}$ wohldefiniert. Ferner ist die Norm submultiplikativ, denn:
	\begin{align*}
		\norm{[b]} \cdot \norm{[c]} &= \underset{x \in I_\lambda}{\inf}\norm{b+x} \cdot \underset{y \in I_\lambda}{\inf}\norm{c+y} = \underset{x,y \in I_\lambda}{\inf}(\norm{b+x} \cdot \norm{c+y}) \geq \\
		&\geq \underset{x,y \in I_\lambda}{\inf}\norm{bc+by+cx+xy} \geq \underset{z \in I_\lambda}{\inf}\norm{bc+z} = \norm{[b]\cdot[c]}
	\end{align*}
Also ist $^\A/_{I_\lambda}$ eine Banachalgebra.

Noch zu zeigen ist $\left({}^\A/_{I_\lambda}\right)^\times = {}^\A/_{I_\lambda} \backslash \{[0]\}$. Hierbei folgend wir dem Beweis zu \cite[Lemma IX.2.5(e)]{Werner2011}.

Ist $[b] \in {}^\A/_{I_\lambda} \backslash \{[0]\}$, dann ist $b \notin I_\lambda$, denn sonst wäre $b-0 = b \in I_\lambda$, also $[b] = [0]$.

Definiere nun $J := \{cb + x ~|~ c\in\A, x\in I_\lambda\}$. Dies ist ein Ideal, welches $I_\lambda$ enthält (ist $x \in I_\lambda$, so ist $x = 0b+x \in J$). Wegen $b \in J \backslash I_\lambda$ ist $I_\lambda \subsetneq J$ und es folgt $J = \A$, da $I_\lambda$ nach \ref{proof:Spektrum-von-a:maxIdeal} ein maximales Ideal ist.

Insbesondere ist daher $e \in J$, es existieren also $c \in A, x \in I_\lambda$, sodass $cb + x = e$. Damit ist $[c] \in {}^\A/_{I_\lambda}$ das Inverse zu $[b]$:
	\[[c][b] = [cb] = [cb + x] = [e]\]

zu \ref{proof:Spektrum-von-a:Projektion}:
Ist $[b] \in {}^\A/_{I_\lambda}$, so gibt es nach \cref{kor:spektrum-nicht-leer} ein $\lambda_b \in \sigma_{^\A/_{I_\lambda}}([b])$. Also ist $\lambda_b[e] - [b] \notin \left({}^\A/_{I_\lambda}\right)^\times$ und daher mit \ref{proof:Spektrum-von-a:AI-BA} bereits $\lambda_b[e] - [b] = [0]$. Dadurch ist $[b] = \lambda_b[e]$ und $\lambda_b \in \CC$ damit eindeutig bestimmt. Wir können daher die folgende Abbildung definieren:
	\[f: \A \to \CC: b \mapsto \lambda_b\]
Diese ist ein Algebrenhomomorphismus, denn für alle $b,c \in \A, \mu \in \CC$ gilt:
\begin{itemize}
	\item[$\bullet$] 
		$\lambda_{\mu b+c}[e] = [\mu b+c] = \mu[b]+[c] = \mu\lambda_b[e] + \mu\lambda_c[e] = (\mu\lambda_b +\lambda_c)[e]$
	
			$\qquad \Rightarrow f(\mu b + c) = \lambda_{\mu b+c} = \mu\lambda_b +\lambda_c  = \mu f(b) + f(c)$
	\item[$\bullet$]
		 $\lambda_{bc}[e] = [bc] = [b][c] = \lambda_b[e]\lambda_c[e] = \lambda_b\lambda_c[e]$
	
			$\qquad f(bc) = \lambda_{bc} = \lambda_b\lambda_c  = f(b)f(c)$
	\item[$\bullet$]
		 $\lambda_e[e] = [e] = 1[e] ~ \Rightarrow ~ f(e) = \lambda_e = 1$
\end{itemize}
Zudem ist $f$ stetig, denn für $b \in \A$ und $\epsilon > 0$ gilt für alle $c \in \A$ mit $\norm{b-c} < \epsilon\norm{[e]}$:
	\begin{align*}
		|f(b)-f(c)| &= |\lambda_b - \lambda_c| = |\lambda_b - \lambda_c|\frac{\norm{[e]}}{\norm{[e]}} = \Big\lVert\lambda_b[e] - \lambda_c[e]\Big\rVert\frac{1}{\norm{[e]}}  = \\
					&= \norm{[b]-[c]}\frac{1}{\norm{[e]}} \leq \norm{b-c}\frac{1}{\norm{[e]}} < \epsilon\norm{[e]}\frac{1}{\norm{[e]}} = \epsilon
	\end{align*}	
Also ist $f \in \SpecC(\A)$. Weiterhin ist $\lambda e - a \in I_\lambda$, also $[0] = [\lambda e - a] \in {}^\A/_{I_\lambda}$ und daher $f(0) = f(\lambda e - a)$. Daraus folgt
	\[0 = f(0) = f(\lambda e - a) = \lambda f(e) - f(a) = \lambda - f(a)\]
und somit ist $f$ gerade der gesuchte Algebrenhomomorphismus mit $f(a) = \lambda$.

Da diese Konstruktion für beliebige $\lambda \in \sigma_\A(a)$ möglich ist, ergibt sich daraus die gesuchte Inklusion:
	\[\sigma_\A(a) \subseteq \SpecC(\A)\]
\end{itemize}
\end{proof}

Hiermit haben wir nun die nötigen Voraussetzungen um zu zeigen, dass $\AlgIso$ isometrisch ist:

\begin{proof}[\Bew{$\AlgIso$ isometrisch}]
Sei $a \in \A$, dann ist $\{f(a) ~|~ f \in \SpecC(\A)\} = \sigma_\A(a)$ (\cref{prop:Spektrum-von-a}) und es gilt:
	\[\underset{f \in \SpecC(\A)}{\sup}\left|f(a)\right| = \underset{\lambda \in \sigma_\A(a)}{\sup}\left|\lambda\right| = R_\A(a)\]
Aus \cref{prop:R-gleich-Norm} wissen wir zudem, dass $R_\A(a) = \norm{a}$ ist, woraus insgesamt folgt, dass $\AlgIso$ isometrisch ist:
	\[\norm{\AlgIso(a)} = \underset{f \in \SpecC(\A)}{\sup}\left|f(a)\right| = R_\A(a) = \norm{a}.\]
\let\qed\relax
\end{proof}

\subsubsection{$\AlgIso$ ist ein \CAlgHom}\label{sec:CAlgHom}

Dass $\AlgIso$ ein Algebrenhomomorphismus ist, haben wir bereits in \Cref{sec:Algebrenhomomorphismus} bewiesen. Damit $\AlgIso$ auch ein \CAlgHom{} ist, fehlt also nur noch die Eigenschaft $\left(\AlgIso(a)\right)^* = \AlgIso(a^*)$. Für diese benötigen wir zwei weitere Hilfsaussagen:

\begin{defn}[selbstadjungiert]\label{defn:selbstadjungiert}
Ein Element $a$ einer \CAlg{} heißt \emph{selbstadjungiert}, wenn gilt:
	\[a^* = a\]
\end{defn}

\begin{prop}\label{prop:Spektrum-reell}
Sei $\A$ eine \CAlg{} und $a \in \A$ selbstadjungiert. Dann ist das Spektrum von $a$ reell, d.h.
	\[\sigma_\A(a) \subseteq \RR\]
\end{prop}

\begin{proof}
Sei $\alpha + i\beta$ ein beliebiges Element von $\sigma_\A(a)$ (mit $\alpha,\beta \in \RR$). Wir wollen zeigen, dass dann $\beta = 0$ gelten muss, also das Element selbst reell ist. Der Beweis hierfür orientiert sich an dem zu \cite[Lemma IX.3.3(c)]{Werner2011}.

Wegen $\alpha +i\beta \in \sigma_\A(a)$ ist	für beliebige $\gamma \in \RR$
	\[(\alpha + i(\beta + \gamma))\cdot e - (a + i\gamma e) = (\alpha+i\beta)\cdot e - a \notin \A^\times\]
und somit $\alpha + i(\beta + \gamma) \in \sigma_\A(a+ i\gamma e)$. Folglich ist mit \Cref{prop:R-gleich-Norm}
	\[|\alpha + i(\beta + \gamma)| \leq \sup\{|\mu| ~|~ \mu \in \sigma_\A(a+ i\gamma e)\} = \norm{a+ i\gamma e}\]
und es gilt (nach \Cref{lemma:CAlg-Eigenschaften} ist $\norm{e} = 1$ und $e^* = e$):
	\begin{align*}
	\alpha^2+\beta^2+2\beta\gamma+\gamma^2 &= \alpha^2 + (\beta+\gamma)^2 = |\alpha + i(\beta + \gamma)|^2 \leq \norm{a+ i\gamma e}^2 = \\
	&= \norm{(a+ i\gamma e)^*(a+ i\gamma e)} = \norm{(a^* - i\gamma e^*)(a+ i\gamma e)} = \\
	&=\norm{(a - i\gamma e)(a+ i\gamma e)} = \norm{a^2 + \gamma^2 e^2} \leq \\
	&\leq \norm{a^2} + |\gamma^2|\norm{e} = \norm{a^2} + \gamma^2
	\end{align*}
Durch Umformen erhalten wir:
	\[2\beta \gamma \leq \norm{a^2} - \alpha^2 - \beta^2\]
Da dies für beliebige $\gamma \in \RR$ gelten soll, muss $\beta$ bereits $0$ sein.

Somit haben wir gezeigt, dass für selbstadjungierte $a \in \A$ Elemente aus $\sigma_\A(a)$ keinen Imaginärteil haben können, d.h. reell sein müssen.
\end{proof}

\begin{prop}\label{prop:Stern-zu-Konjugation}
Ist $\A$ eine \CAlg{} und $f: \A \to \CC$ ein Algebrenhomomorphismus, so gilt:
	\[f(a^*) = \overline{f(a)}\]
\end{prop}

\begin{proof}
Sei $a \in \A$. Dann können wir $a^*$ wie folgt zerlegen:
	\[a^* = \frac{1}{2}\Big((a^* + a) - ii(a^* - a)\Big)\]
Dabei sind sowohl $(a^* + a)$ als auch $i(a^* - a)$ selbstadjungiert, denn es gilt
	\[(a^* + a)^* = a^{**} + a^* = a + a^* = a^* + a\]
und
	\[(i(a^* - a))^* = \overline{i}(a^{**} - a^*) = -i(a - a^*) = i(a^* - a)\]
Also ist:
	\[f(a^* + a) \overset{\ref{prop:Spektrum-von-a}}{\in} \sigma_\A(a^* + a) \overset{\ref{prop:Spektrum-reell}}{\subseteq} \RR\]
und analog $f(i(a^* - a)) \in \RR$. Daraus folgt (da $f$ ein Algebrenhomomorphismus ist):
	\begin{align*}
	f(a^*) &= \frac{1}{2}f(2a^*) = \frac{1}{2}f\Big((a^* + a) - ii(a^* - a)\Big) = \frac{1}{2}\Big(f(a^* + a) - if(i(a^* - a))\Big) = \\
				&= \frac{1}{2}\left(\overline{f(a^* + a) + if(i(a^* - a))}\right) = \frac{1}{2}\overline{f\left((a^* + a) + ii(a^* - a)\right)} = \\
				&= \frac{1}{2}\overline{f(a^* + a - a^* + a)} = \frac{1}{2}\overline{f(2a)} = \overline{f(a)} \qedhere
	\end{align*}
\end{proof}

Mit Hilfe dieser Aussage können wir nun zeigen, dass $\AlgIso$ ein \CAlgHom{} ist:

\begin{proof}[\Bew{$\AlgIso$ \CAlgHom}]Für alle $a \in \A$ und $f \in \SpecC(\A)$ gilt:
	\[\Big(\AlgIso(a)\Big)^*(f) = (\tau_a^*)(f) = \overline{\tau_a(f)} = \overline{f(a)} \overset{\ref{prop:Stern-zu-Konjugation}}{=} f(a^*) = \tau_{a^*}(f) = \AlgIso(a^*)(f) \]
	Damit ist $\AlgIso$ ein \CAlgHom.
\let\qed\relax
\end{proof}

\subsubsection{$\AlgIso$ ist bijektiv}\label{sec:Bijektiv}

Als letzte Eigenschaft bleibt noch zu zeigen, dass $\AlgIso$ auch bijektiv ist.

\begin{proof}[\Bew{$\AlgIso$ bijektiv}]\ 

\begin{itemize}
\item $\AlgIso$ ist injektiv, denn:

Sind $a, b \in \A$, sodass $\AlgIso(a) = \AlgIso(b)$, dann folgt (da $\AlgIso$ ein Algebrenhomomorphismus und daher linear ist):
	\[0 = \norm{\AlgIso(a) - \AlgIso(b)} = \norm{\AlgIso(a-b)} = \norm{a-b}\]
und damit $a = b$.

\item $\AlgIso$ ist surjektiv, denn:

Wir wollen hier den Satz von Stone-Weiherstraße (\ref{satz:SW}) anwenden. Zunächst ist leicht zu sehen, dass $\AlgIso(\A) \subseteq \stetig(\SpecC(\A))$ eine Unteralgebra ist ($\AlgIso$ ist ein Algebrenhomomorphismus und daher $\AlgIso(\A)$ abgeschlossen unter den Algebraoperationen). Ferner gilt:
\begin{itemize}
	\item In \cref{sec:Algebrenhomomorphismus} haben wir bereits gezeigt, dass $\AlgIso(e)$ die konstante Einsabbildung ist (das Einselement in $\stetig(\SpecC(\A))$. Damit sind wegen $\AlgIso(\lambda e) = \lambda \AlgIso(e)$ für $\lambda \in \CC$ bereits alle konstanten Funktionen in $\AlgIso(\A)$ enthalten.
	\item Sind $f, g \in \SpecC(\A)$ mit $f \neq g$, so gibt es also ein $a \in \A$ mit $f(a) \neq g(a)$. Damit gilt:
		\[\AlgIso(a)(f) = \tau_a(f) = f(a) \neq g(a) = \tau_a(g) = \AlgIso(a)(g)\]
		Also ist $\tau_a \in \AlgIso(\A)$ eine Funktion, die $f$ und $g$ trennt.
	\item Ist $\tau_a \in \AlgIso(\A)$, so auch $\tau_{a^*}$, und es folgt nach \cref{sec:CAlgHom}:
		\[\overline{\tau_a} = (\AlgIso(a))^* = \AlgIso(a^*) = \tau_{a^*} \in \AlgIso(\A)\]
\end{itemize}
Also liegt dem Satz von Stone-Weiherstraß zufolge $\AlgIso(\A)$ dicht in $\stetig(\SpecC(\A))$, d.h. $\overline{\AlgIso(\A)} = \stetig(\SpecC(\A))$.

Gleichzeitig ist $\AlgIso(\A)$ aber abgeschlossen in $\stetig(\SpecC(\A))$. Ist nämlich $(\AlgIso(a_n))_{n\in\NN}$ eine Folge in $\AlgIso(\A)$, die in $\stetig(\SpecC(\A))$ konvergiert, so konvergiert auch die Folge $(a_n)_{n\in\NN}$ in $\A$ (da $\AlgIso$ isometrisch und ein Algebrenhomomorphismus ist). Aufgrund der Vollständigkeit von $\A$ hat diese Folge einen Grenzwert $a \in \A$ und $\AlgIso(a)$ ist - wieder aufgrund der Isometrie von $\AlgIso$ - der Grenzwert von $\AlgIso(a_n)$ in $\AlgIso(\A)$. 

Es ist folglich insgesamt
	\[\stetig(\SpecC(\A)) = \overline{\AlgIso(\A)} = \AlgIso(\A)\]
und somit ist $\AlgIso$ surjektiv.
\end{itemize}

Hiermit ist der Satz von Gelfand-Neumark nun vollständig bewiesen. Die Abbildung $\AlgIso: \A \to \stetig(\SpecC(\A)): a \mapsto \left(\tau_a: f \mapsto f\left(a\right)\right)$ ist also tatsächlich ein isometrischer Isomorphismus von \CAlgn. Er zeigt damit auch, dass tatsächlich jede \CAlg{} bis auf isometrische Isomorphie eine \CAlg{} der stetigen komplexwertigen Funktionen über einem kompakten Hausdorffraum ist.
\end{proof}