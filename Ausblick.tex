\phantomsection
\addcontentsline{toc}{section}{Ausblick}
\section*{Ausblick}
\begin{center}
Sind nun alle Fragen beantwortet?
\end{center}

Nun, hoffentlich zumindest die in der Einleitung aufgeworfenen. Wir haben gesehen, dass \CAlgn{} als Räume stetiger komplexwertiger Funktionen über kompakten Hausdorffräumen aufgefasst werden können, haben gezeigt, dass die Kategorie der \CAlgn{} dual zu der der kompakten Hausdorffräume ist, und haben schließlich die zur Kategorie der \komTopMann{} duale Unterkategorie der \CAlgn{} gefunden. Aber wie so oft in der Mathematik bringt jede Antwort auch wieder neue Fragen mit sich.

So könnte man sich nun fragen, ob sich die \CAlgMann{} noch irgendwie \glqq schöner\grqq{} charakterisieren lassen. Etwa durch eine zusätzliche innere Struktur analog zur Involution, die die Struktur ist, deren Existenz die \CAlgn{} innerhalb der Banachalgebren charakterisiert. Auch wäre es interessant zu untersuchen, ob \CAlgMann{} irgendwelche besonderen Eigenschaften haben, die allgemeine \CAlgn{} nicht notwendigerweise besitzen.

Eine andere Frage könnte sein, was passiert, wenn wir eine andere Unterkategorie von $\KatTop$ bzw. $\KatTopMan$ betrachten - etwa die der (kompakten) differenzierbaren Mannigfaltigkeiten. Jede solche ist auch eine topologische Mannigfaltikeit, hat also eine duale \CAlgMan{}. Aber kann man einer \CAlgMan{} irgendwie ansehen, dass sie dual zu einer differenzierbaren Mannigfaltigkeit ist? Und lässt sich hierfür wiederum eine passende Erweiterung der Gelfand-Dualität finden? 

Gerade die letzte Frage dürfte vermutlich etwas aufwändiger zu beantworten sein als der Schritt von kompakten Hausdorffräumen zu topologischen Mannigfaltigkeiten. Denn im Gegensatz zu topologoischen Mannigfaltigkeiten sind zwei differenzierbare Mannigfaltigkeiten nicht automatisch \glqq gleich\grqq{} (diffeomorph), wenn sie es nur als topologische Räume sind\footnote{Ein berühmtes Beispiel hierfür sind die exotischen $S^7$ - siehe \cite{Elwes2011} für eine kurze Einführung und \cite{Bognat2011} für einen ausführlichen Beweis}. Hier müsste also erst eine zusätzliche Struktur auf \CAlgn{} definiert werden, anhand der sich auch eigentlich isomorphe \CAlgn{} in solchen Fällen unterscheiden lassen (also wenn sie zu homöomorphen, aber nicht diffeomorphen Mannigfaltigkeit \glqq gehören\grqq). Die Frage ist also:

\begin{center}
Was ist der Unterschied zwischen $\stetig(S^7)$ und $\stetig(S^7)$?
\end{center}