\subsubsection{$\AlgIso$ ist isometrisch}\label{sec:isometrisch}

Als nächsten Schritt werden wir zeigen, dass $\AlgIso$ isometrisch ist, d.h dass für alle $a \in \A$ gilt:
	\[\norm{a} = \norm{\AlgIso(a)} := \underset{f \in \SpecC(\A)}{\sup}\left|f(a)\right|\]
Dazu werden wir die beiden Hilfsaussagen $\norm{a} = R_\A(a) := \sup\{|\lambda| ~|~ \lambda \in \sigma_\A(a)\}$ und $\sigma_\A(a) = \{f(a) ~|~ f \in \SpecC(\A)\}$ verwenden, die wir in den folgenden Propositionen zeigen werden. Einige der Aussagen werden wir später noch einmal benötigen und sie daher allgemeiner für Banachalgebren beweisen.

\begin{defn}[Spektrum] %Folgt Def 2.1.11
Sei $\A$ eine Banachalgebra und $a \in \A$. Dann heißt die Menge
	\[\sigma_\A(a) := \{\lambda \in \CC ~|~ \lambda e - a \notin \A^\times\}\]
\emph{Spektrum von $a$} und die Zahl
	\[R_\A(a) := \sup\{|\lambda| ~|~ \lambda \in \sigma_\A(a)\}\]
\emph{Spektralradius von $a$}. 
\end{defn}


\begin{prop}\label{prop:Konvergenz}
Ist $(\alpha_n)_{n\in\NN}$ eine Folge in $\RR$, die gegen ein $\alpha \in [0,1[$ konvergiert, so konvergiert $(\alpha_n)^n$ gegen $0$
\end{prop}

\begin{proof}
Zunächst gibt es wegen $\alpha_n \xlongrightarrow{n\to\infty} \alpha$ ein $K \in \NN$, sodass für alle $n\geq K$ gilt:
	\[|\alpha_n - \alpha| < \frac{1-\alpha}{2}.\]
Außerdem ist $\frac{1+\alpha}{2} < \frac{1+1}{2} = 1$ und daher konvergiert $\left(\frac{1+\alpha}{2}\right)^n$ gegen $0$. Also existiert für jedes $\epsilon > 0$ ein $N \in \NN$, sodass
	\[\forall n \geq N: \left(\frac{1+\alpha}{2}\right)^n < \epsilon.\]
Ohne Einschränkung können wir dabei $N \geq K$ wählen. Damit gilt für alle $n \geq N$:
	\[\left|\left(\alpha_n\right)^n\right| = \left|\alpha_n\right|^n \leq \left(\left|\alpha_n - \alpha\right| + \left|\alpha\right|\right)^n \leq \left(\frac{1-\alpha}{2} + \alpha\right)^n = \left(\frac{1+\alpha}{2}\right)^n < \epsilon\]
Also konvergiert $(\alpha_n)^n$ gegen $0$.
\end{proof}


\begin{prop}\label{prop:R-groesser-lima}
Sei $\A$ eine Banach-Algebra und $a \in \A$. Dann gilt:
\[ R_\A(a) \geq \lima := \underset{k\to\infty}{\lim}\norm{a^k}^\frac{1}{k}\]
\end{prop}

\begin{proof}Zunächst zum Sonderfall $a = 0$: Es ist $0 e - a = 0 - 0 = 0 \notin \A^\times$, also $0 \in \sigma_\A(a)$. Folglich gilt:
	\[R_\A(a) = \sup\{|\lambda| ~|~ \lambda \in \sigma_\A(a)\} \geq 0 = \underset{k\to\infty}{\lim}\norm{0^k}^\frac{1}{k} = \underset{k\to\infty}{\lim}\norm{a^k}^\frac{1}{k}\]

Im Folgenden können wir uns daher auf $a \neq 0$ beschränken. Hierfür zeigen wir zunächst,
\begin{proofenum}
	\item \label{proof:R-groesser-lima:limaexists}
		$\lim_{k\to\infty}\norm{a^k}^\frac{1}{k}$ existiert und $L(a) \leq \norm{a^n}^\frac{1}{n}$ gilt für alle $n \in \NN$.
\setcounter{temp}{\value{proofenumi}}
\end{proofenum}
Dann nehmen wir an es wäre $R_\A(a) < \lima$, d.h. $\lambda e - a$ wäre für alle $\lambda \in \CC$ mit $|\lambda| \geq \lima$ invertierbar. Für alle solche $\lambda$, beliebige $n \in \NN$ und $\omega := \exp{\frac{2\pi i}{n}} \in \CC$ zeigen wir dann die folgenden Aussagen: 
\begin{proofenum}
\setcounter{proofenumi}{\value{temp}}
	\item \label{proof:R-groesser-lima:invertierbar}
		$\frac{1}{n}\sum_{k=1}^n \left(e-\frac{\omega^k a}{\lambda}\right)^{-1} \in \A$ ist das Inverse von $e - \frac{a^n}{\lambda^n}$.
	\item \label{proof:R-groesser-lima:abschaetzung}
		Es gibt ein von $\lambda$ unabhängiges $r_\textup{konst} \in \RR$, sodass folgende Abschätzung gilt:
		\[\norm{\left(e - \frac{a^n}{\lima^n}\right)^{-1} - \left(e - \frac{a^n}{\lambda^n}\right)^{-1}} \leq r_\textup{konst}\left|\lima - \lambda\right|\]
	\item \label{proof:R-groesser-lima:nach0}
		Es gilt $\left(\frac{\norm{a^n}^\frac{1}{n}}{\lima}\right)^n = \frac{\norm{a^n}}{\lima^n} \xlongrightarrow{n\to\infty} 0$
\end{proofenum}
Dann ist jedoch \ref{proof:R-groesser-lima:nach0} ein Widerspruch zu \ref{proof:R-groesser-lima:limaexists}, die Annahme also falsch und es folgt die zu zeigende Aussage.

Zu \ref{proof:R-groesser-lima:limaexists}: Hierzu zeigen wir, dass die $\norm{a^k}^\frac{1}{k}$ eine monoton fallende Folge bilden. Da diese Folge außerdem durch $0$ nach unten beschränkt ist, muss der Limes damit bereits existieren.
\begin{align*}
	\norm{a^k}^\frac{1}{k} &= \norm{a^k}^\frac{1}{k}\cdot\norm{a}^{-\frac{1}{k+1}+\frac{1}{k+1}} = \norm{a^k}^\frac{1}{k}\cdot\left(\norm{a}^k\right)^{-\frac{1}{k(k+1)}}\cdot\norm{a}^{\frac{1}{k+1}} \geq \\
	&\geq \norm{a^k}^\frac{1}{k}\cdot\norm{a^k}^{-\frac{1}{k(k+1)}}\cdot\norm{a}^{\frac{1}{k+1}} = \norm{a^k}^{\frac{k+1-1}{k(k+1)}}\cdot\norm{a}^{\frac{1}{k+1}} = \left(\norm{a^k}\norm{a}\right)^{\frac{1}{k+1}} \geq \\
	&\geq \norm{a^{k+1}}^\frac{1}{k+1}
	\end{align*}
Zusätzlich ergibt sich hieraus für alle $n \in \NN$ die Abschätzung $\lima = \underset{k\to\infty}{\lim}\norm{a^k}^\frac{1}{k} \leq \norm{a^n}^\frac{1}{n}$.

Für die folgenden Schritte nehmen wir nun an, $\lambda e - a$ wäre für alle $|\lambda| \geq \lima$ invertierbar.

zu \ref{proof:R-groesser-lima:invertierbar}:
Seien $|\lambda| \geq \lima$ und $n,k \in \NN$ beliebig, dann ist $\left|\frac{\lambda}{\omega^k}\right| = \frac{|\lambda|}{1} \geq \lima$, also $e-\frac{\omega^k a}{\lambda} = \frac{\omega^k}{\lambda} \left(\frac{\lambda}{\omega^k}e - a\right) \in \A^\times$ nach Annahme. Somit existiert $\frac{1}{n}\sum_{k=1}^n \left(e-\frac{\omega^k a}{\lambda}\right)^{-1} \in \A$.

Wegen $\left(\omega^{k}\right)^n = \exp{\frac{2\pi i kn}{n}} = \exp{2\pi i k} = 1$ ist
\begin{align*}
	e - \frac{a^n}{\lambda^n} &= \left(\frac{\omega^k a}{\lambda}\right)^0 - \left(\frac{\omega^k a}{\lambda}\right)^n = \sum_{l=1}^n \left[\left(\frac{\omega^k a}{\lambda}\right)^{l-1} - \left(\frac{\omega^k a}{\lambda}\right)^l\right] = \\
		&=\left(e - \frac{\omega^k a}{\lambda}\right)\sum_{l=1}^n \left(\frac{\omega^k a}{\lambda}\right)^{l-1}. \quad (\#)
\end{align*}
Ferner gilt $\sum_{k=1}^n \left(\omega^{l-1}\right)^k = \begin{cases}n &,l=1 \\ 0 &,l\in\{2, \dots, n\}\end{cases}$ ($\sim$), denn (vgl. \cite[S. 10-8,10-9]{Einheitswurzeln}):

Ist $l\in\{2, \dots, n\}$, so ist $\omega^{l-1} = \exp{\frac{2\pi i(l-1)}{n}} \neq 1$ und $\left(\omega^{l-1}\right)^n = 1$. Es folgt
	\[0 = \left(\omega^{l-1}\right)^n - 1 = \left(\omega^{l-1} - 1\right)\left(\left(\omega^{l-1}\right)^{n-1} + \dots + \omega^{l-1} + 1\right)\]
und damit
	\[0 = \left(1 + \left(\omega^{l-1}\right)^{n-1} + \dots + \omega^{l-1}\right) = \left(\left(\omega^{l-1}\right)^{n} + \left(\omega^{l-1}\right)^{n-1} + \dots + \omega^{l-1}\right) = \sum_{k=1}^n \left(\omega^{l-1}\right)^k.\]

Zusammen ergibt das die Behauptung aus \ref{proof:R-groesser-lima:invertierbar}:
\begin{align*}
	\left(\frac{1}{n}\sum_{k=1}^n \left(e-\frac{\omega^k a}{\lambda}\right)^{-1}\right) \left(e - \frac{a^n}{\lambda^n}\right) \overset{(\#)}{=} \left(\frac{1}{n}\sum_{k=1}^n \left(e-\frac{\omega^k a}{\lambda}\right)^{-1}\right) \left(e - \frac{\omega^k a}{\lambda}\right)\sum_{l=1}^n \left(\frac{\omega^k a}{\lambda}\right)^{l-1} = \\
	= \frac{1}{n}\sum_{k=1}^n \sum_{l=1}^n \left(\frac{\omega^k a}{\lambda}\right)^{l-1} = \frac{1}{n} \sum_{l=1}^n \left(\frac{a}{\lambda}\right)^{l-1} \sum_{k=1}^n \left(\omega^{l-1}\right)^k \overset{(\sim)}{=} \frac{1}{n} \left(\frac{a}{\lambda}\right)^{1-1}n = e
\end{align*}

zu \ref{proof:R-groesser-lima:abschaetzung}: Seien wieder $|\lambda| \geq \lima$ und $n,k \in \NN$ beliebig, dann gilt:

\begin{align*}
\left(\frac{\lima}{\omega^k}e-a\right)^{-1} 
	\left(-\lima\right. &+ \left.\lambda\right)\frac{a}{\omega^k} 
	\left(\frac{\lambda}{\omega^k}e - a\right)^{-1} = \\
= \left(e-\frac{\omega^k}{\lima}a\right)^{-1} 
	\frac{\omega^k}{\lima} \left(-\lima\right. &+ \left.\lambda\right)\frac{a}{\omega^k} \frac{\omega^k}{\lambda}
	\left(e - \frac{\omega^k}{\lambda}a\right)^{-1} = \\
= \left(e-\frac{\omega^k}{\lima}a\right)^{-1} 
	\left(-\frac{\omega^k}{\lambda}a \right.&+\left. \frac{\omega^k}{\lima}a\right)
	\left(e - \frac{\omega^k}{\lambda}a\right)^{-1} = \\
= \left(e-\frac{\omega^k}{\lima}a\right)^{-1} 
	\left(\left(e -\frac{\omega^k}{\lambda}a\right) \right.&-\left. \left(e - \frac{\omega^k}{\lima}a\right)\right)
	\left(e - \frac{\omega^k}{\lambda}a\right)^{-1} = \\	
= \left(e-\frac{\omega^k}{\lima}a\right)^{-1} 
	\Bigg(e &- \left(e - \frac{\omega^k}{\lima}a\right)\left(e - \frac{\omega^k}{\lambda}a\right)^{-1}\Bigg)
	 = \\
= \left(e-\frac{\omega^k}{\lima}a\right)^{-1} &- \left(e - \frac{\omega^k}{\lambda}a\right)^{-1} \quad (\star)
\end{align*}

Daraus folgt mit Hilfe von $\left(e - \frac{a^n}{\lambda^n}\right)^{-1} = \frac{1}{n}\sum_{k=1}^n \left(e-\frac{\omega^k a}{\lambda}\right)^{-1}$ aus \ref{proof:R-groesser-lima:invertierbar}:
\begin{align*}
\norm{\left(e-\frac{a^n}{\lima^n}\right)^{-1} - \left(e - \frac{a^n}{\lambda^n}\right)^{-1}} 
	\overset{\ref{proof:R-groesser-lima:invertierbar}}{=} \norm{\frac{1}{n}\sum_{k=1}^n\left(e-\frac{\omega^k}{\lima}a\right)^{-1} - \frac{1}{n}\sum_{k=1}^n\left(e - \frac{\omega^k}{\lambda}a\right)^{-1}} \leq \\
	\leq \frac{1}{n}\sum_{k=1}^n \norm{\left(e-\frac{\omega^k}{\lima}a\right)^{-1} - \left(e - \frac{\omega^k}{\lambda}a\right)^{-1}} = \\
	\overset{(\star)}{=} \frac{1}{n}\sum_{k=1}^n \norm{\left(-\lima + \lambda\right)\frac{a}{\omega^k} \left(\frac{\lima}{\omega^k}e-a\right)^{-1} \left(\frac{\lambda}{\omega^k}e - a\right)^{-1}} \leq \\
	\leq \frac{1}{n}\sum_{k=1}^n \left|-\lima + \lambda\right| \cdot \frac{\norm{a}}{\left|\omega^k\right|} \cdot \left(\underset{|\lambda| \geq \lima}{\sup}\norm{(\lambda e - a)^{-1}}\right)^2 = \\
	= \left|\lima - \lambda\right| \cdot \norm{a} \cdot \underset{|\lambda| \geq \lima}{\sup}\norm{(\lambda e - a)^{-1}}^2	
\end{align*}

Falls $\sup_{|\lambda| \geq \lima}\norm{(\lambda e - a)^{-1}}$ endlich ist, dann wäre die Aussage \ref{proof:R-groesser-lima:abschaetzung} hiermit gezeigt (setze $r_\textup{konst} := \norm{a} \cdot \sup_{|\lambda| \geq \lima}\norm{(\lambda e - a)^{-1}}^2$).

Für $|\lambda| \geq 2\norm{a}$ ist $\norm{\frac{a}{\lambda}} = \frac{\norm{a}}{|\lambda|} \leq \frac{\norm{a}}{2\norm{a}} = \frac{1}{2}$ und daher $\sum_{k=0}^\infty \left(\frac{a}{\lambda}\right)^k$ absolut konvergent. Außerdem gilt
	\[(\lambda e - a)\cdot \frac{1}{\lambda}\sum_{k=0}^\infty \left(\frac{a}{\lambda}\right)^k = \sum_{k=0}^\infty \left(\frac{a}{\lambda}\right)^k - \sum_{k=0}^\infty \left(\frac{a}{\lambda}\right)^{k+1} = 1\]
und somit die Abschätzung
	\[\norm{(\lambda e - a)^{-1}} = \norm{\frac{1}{\lambda}\sum_{k=0}^\infty \left(\frac{a}{\lambda}\right)^k} \leq \frac{1}{|\lambda|}\sum_{k=0}^\infty\norm{\frac{a}{\lambda}}^k \leq \frac{1}{2\norm{a}}\sum_{k=0}^\infty\left|\frac{1}{2}\right|^k = \frac{1}{\norm{a}}. \]

Also ist $(\lambda e - a)^{-1}$ auf $\{\lambda \in \CC ~|~ |\lambda| \geq 2\norm{a}\}$ durch $\frac{1}{\norm{a}}$ beschränkt.

Da die Abbildung $\lambda \mapsto (\lambda e - a)^{-1}$ stetig ist(als Verknüpfung der stetigen Abbildungen Skalarmultiplikation, Addition und Inversenbildung (siehe \Cref{lemma:BAlg-Eigenschaften})), ist sie auch auf dem verbleibenden Kompaktum $\{\lambda \in \CC ~|~ 2\norm{a} \geq |\lambda| \geq \lima\}$ beschränkt. 

Somit ist $\sup_{|\lambda| \geq \lima}\norm{(\lambda e - a)^{-1}}$ tatsächlich endlich.


zu \ref{proof:R-groesser-lima:nach0}:
Aufgrund der Stetigkeit von Inversenbildung und Addition (\Cref{lemma:BAlg-Eigenschaften}) gilt für eine Folge $(a_n)_{n\in\NN} \subset \A$ mit $e - a_n \in \A^\times$ für alle $n$:
	\[a_n \xlongrightarrow{n\to\infty} 0 \iff e - a_n  \xlongrightarrow{n\to\infty} e \iff (e-a_n)^{-1} \xlongrightarrow{n\to\infty} e\]

Für beliebiges $l \in \NN$ ist nun 
	\[\underset{n\to \infty}{\lim}\frac{\norm{a^n}^\frac{1}{n}}{\lima+\frac{1}{l}} = \frac{\underset{n\to \infty}{\lim}\norm{a^n}^\frac{1}{n}}{\underset{k\to\infty}{\lim}\norm{a^k}^\frac{1}{k}+\frac{1}{l}} < 1\]
und es folgt mit \Cref{prop:Konvergenz}
	\[\norm{\frac{a^n}{\left(\lima+\frac{1}{l}\right)^n}} = \frac{\norm{a^n}}{\left|\lima+\frac{1}{l}\right|^n} = \left(\frac{\norm{a^n}^\frac{1}{n}}{\lima+\frac{1}{l}}\right)^n \xlongrightarrow{n\to\infty} 0.\]
Also gilt auch
	\[\frac{a^n}{\left(\lima+\frac{1}{l}\right)^n} \xlongrightarrow{n\to\infty} 0 \qquad \text{und damit} \qquad \left(e - \frac{a^n}{(\lima+\frac{1}{l})^n}\right)^{-1} \xlongrightarrow{n\to\infty} e.\]

Ferner haben wir aus \ref{proof:R-groesser-lima:abschaetzung} die Abschätzung:
	\[\norm{\left(e - \frac{a^n}{\lima^n}\right)^{-1} - \left(e - \frac{a^n}{(\lima + \frac{1}{l})^n}\right)^{-1}} \leq r_\textup{konst}\left|\lima-\left(\lima+\frac{1}{l}\right)\right| = \frac{r_\textup{konst}}{l}\]
Zusammen ergibt das
	\begin{align*}\frac{r_\textup{konst}}{l} &\geq \underset{n\to\infty}{\lim} \norm{\left(e - \frac{a^n}{\lima^n}\right)^{-1} - \left(e - \frac{a^n}{(\lima + \frac{1}{l})^n}\right)^{-1}} = \\
	&= \underset{n\to\infty}{\lim} \norm{\left(e - \frac{a^n}{\lima^n}\right)^{-1} - e}.
	\end{align*}
Da dies für beliebige $l \in \NN$ gilt, folgt
	\[\underset{n\to\infty}{\lim} \norm{\left(e - \frac{a^n}{\lima^n}\right)^{-1} - e} = 0\]
und hieraus
	\[\left(e - \frac{a^n}{\lima^n}\right)^{-1} \xlongrightarrow{n\to\infty} e, \text{ also } \frac{a^n}{\lima^n} \xlongrightarrow{n\to\infty} 0 \text{ und } \frac{\norm{a^n}}{\lima^n} \xlongrightarrow{n\to\infty} 0\]
was die Behauptung \ref{proof:R-groesser-lima:nach0} ist.	

Daraus jedoch entsteht ein Widerspruch, denn aus \ref{proof:R-groesser-lima:limaexists} ergibt sich: 
	\[\frac{\norm{a^n}}{\lima^n} \geq \frac{\norm{a^n}}{\norm{a^n}} = 1 \text{ für alle } n\in \NN\]
Somit muss die Annahme falsch gewesen sein und es gibt doch ein $\lambda \in \CC$ mit $|\lambda| \geq \lima$ und $\lambda e - a \notin \A^\times$, d.h. $\lambda \in \sigma_\A(a)$. Daher gilt die Aussage der Proposition:
	\[R_\A(a) = \sup\{|\lambda| ~|~ \lambda \in \sigma_\A(a)\} \geq \lima \qedhere\]
\end{proof}


\begin{kor}\label{kor:spektrum-nicht-leer}
Sei $\A$ eine Banach-Algebra, $a \in \A$. Dann ist $\sigma_\A(a) \neq \emptyset$.
\end{kor}

\begin{proof}
Im Beweis zu \Cref{prop:R-groesser-lima} haben wir gezeigt, dass es ein $\lambda \in \sigma_\A(a)$ gibt, welches größer als $\lima$ ist. Damit ist $\sigma_\A(a)$ insbesondere auch nicht-leer.
\end{proof}

\begin{prop}\label{prop:R-gleich-Norm}
Ist $\A$ eine \CAlg{} und $a \in \A$, so gilt: 
	\[R_\A(a) = \norm{a}\]
\end{prop}

\begin{proof}

\begin{itemize}
	\item[\glqq$\geq$\grqq] Aus \Cref{prop:R-groesser-lima} wissen wir bereits, dass $R_\A(a) \geq \lima = \underset{k\to\infty}{\lim}\norm{a^k}^\frac{1}{k}$ gilt. Also genügt es zu zeigen, dass $\underset{k\to\infty}{\lim}\norm{a^k}^\frac{1}{k} = \norm{a}$ ist.
	
Zunächst gilt $\norm{a^{2^k}}^\frac{1}{2^k} = \norm{a}$, denn für $k \in \NN$ ist
	\begin{align*}
	\norm{(a^*a)^{2^k}} &= \norm{(a^*a)^{2^{k-1}}(a^*a)^{2^{k-1}}} = \norm{((a^*a)^*)^{2^{k-1}}(a^*a)^{2^{k-1}}} = \\
	&= \norm{\left((a^*a)^{2^{k-1}}\right)^*(a^*a)^{2^{k-1}}} = \norm{(a^*a)^{2^{k-1}}}^2.
	\end{align*}
und damit (unter wiederholter Anwendung dieser Aussage)
	\begin{align*}
	\norm{a^{2^k}}^2 &= \norm{\left(a^{2^k}\right)^*a^{2^k}} = \norm{(a^*a)^{2^k}} = \norm{(a^*a)^{2^{k-1}}}^2 = \dots = \norm{(a^*a)^{2^{0}}}^{2^k} = \\
	&= \norm{a^*a}^{2^k} = \left(\norm{a}^2\right)^{2^k} = \norm{a}^{2^{k+1}}. \quad (\#)
	\end{align*}	
	
Nachdem wir im Beweis zu \Cref{prop:R-groesser-lima} in \ref{proof:R-groesser-lima:limaexists} bereits gezeigt haben, dass die $\norm{a^k}^\frac{1}{k}$ eine absteigende Folge bilden, ergibt sich hieraus
	\[\norm{a} \overset{(\#)}{=} \norm{a^{2^k}}^\frac{1}{2^k} \overset{\ref*{prop:R-groesser-lima}\ref*{proof:R-groesser-lima:limaexists}}{\leq} \norm{a^{k}}^\frac{1}{k} \overset{\ref*{prop:R-groesser-lima}\ref*{proof:R-groesser-lima:limaexists}}{\leq} \norm{a^1}^1 = \norm{a}\]
und folglich gilt
	\[R_\A(a) \geq \underset{k\to\infty}{\lim}\norm{a^k}^\frac{1}{k} = \underset{k\to\infty}{\lim}\norm{a} = \norm{a}.\]	
	
	
	\item[\glqq$\leq$\grqq] Sei $\lambda \in \CC$ mit $|\lambda| > \norm{a}$. Wir wollen zeigen, dass dann $\lambda e - a$ invertierbar ist.

Zunächst ist
	\[\norm{\frac{a}{\lambda}} = \frac{\norm{a}}{|\lambda|} < 1\]
Also konvergiert die folgende geometrische Reihen absolut
	\[\frac{1}{\lambda}\sum_{m=0}^\infty\left(\frac{a}{\lambda}\right)^m = \sum_{m=0}^\infty\left(\frac{a^m}{\lambda^{m+1}}\right) \]
und ist das Inverse zu $\lambda e - a$, denn:
	\[(\lambda e - a)\cdot \sum_{m=0}^\infty\frac{a^m}{\lambda^{m+1}} = \sum_{m=0}^\infty\left(\frac{a^m}{\lambda^{m}}\right) - \sum_{m=0}^\infty\left(\frac{a^{m+1}}{\lambda^{m+1}}\right) = \frac{a^0}{\lambda^0} +  \sum_{m=1}^\infty\left(\frac{a^m}{\lambda^{m}}\right) - \sum_{m=1}^\infty\left(\frac{a^m}{\lambda^{m}}\right) = e\]
	
Insgesamt ist damit gezeigt: Für alle $\lambda \in \CC$ folgt aus $|\lambda| > \norm{a}$, dass $\lambda e - a$ invertierbar ist und daher $\lambda$ nicht in $\sigma_\A(a)$ liegt. Daraus ergibt sich die Aussage:
	\[R_\A(a) = \sup\{|\lambda| ~|~ \lambda \in \sigma_\A(a)\} \leq \norm{a} \qedhere\]
\end{itemize}	
\end{proof}


Für die nächste Proposition benötigen wir die folgende Definition:

\begin{defn}[Ideal] 
Sei $\A$ eine Banachalgebra. Dann heißt eine Teilmenge $I \subseteq \A$ ein \emph{Ideal von $\A$}, wenn gilt:
\begin{itemize}
	\item $x,y \in I \Rightarrow x+y \in I$
	\item $x \in I, a \in \A \Rightarrow a \cdot x \in I$
\end{itemize}
Ist zusätzlich
\begin{itemize}
	\item $I \subsetneq \A$,
\end{itemize}
so heißt $I$ \emph{echtes Ideal}. Gilt außerdem noch
\begin{itemize}
	\item $\forall J \subseteq \A \text{ Ideal}: I \subsetneq J \Rightarrow J = \A,$
\end{itemize}
dann ist $I$ ein \emph{maximales Ideal}.
\end{defn}

\begin{bem}
Da Banachalgebren hier immer ein Einselement besitzen, folgt aus den beiden genannten Eigenschaften zusätzlich die Abgeschlossenheit eines Ideals unter Skalarmultiplikation. Ist nämlich $\lambda \in \CC$ und $x \in I \subseteq \A$, so ist $\lambda e \in \A$ und daher $\lambda x = (\lambda e) x \in I$. Also sind Ideale insbesondere Untervektorräume.
\end{bem}

\begin{prop}\label{prop:Spektrum-von-a}
Sei $\A$ eine \CAlg, $a \in \A$. Dann gilt:
\[ \{f(a) ~|~ f \in \SpecC(\A)\} = \sigma_\A(a)\]
\end{prop}

\begin{proof}
\begin{itemize}
\item[\glqq$\subseteq$\grqq]%Folgt Bem 2.1.25
 \Ann $\exists f \in \SpecC(\A): f(a) \notin \sigma_\A(a)$
	
Dann ist $f(a)\cdot e - a$ invertierbar, d.h. es gibt ein $b \in \A$, sodass $(f(a)\cdot e - a)\cdot b =e$. Damit aber folgt der Widerspruch:
	\[1 = f(e) = f((f(a)\cdot e - a)\cdot b) = (f(a)\cdot f(e) - f(a))\cdot f(b) = 0\cdot f(b) = 0\]
	
\item[\glqq$\supseteq$\grqq]%Folgt Lemma 2.1.30
Sei $\lambda \in \sigma_\A(a)$ beliebig, dann suchen wir ein $f \in \SpecC(\A)$ mit $f(a) = \lambda$. Die Konstruktion dieses Elements erfolgt in mehreren Schritten:
\begin{proofenum}
	\item \label{proof:Spektrum-von-a:maxIdeal}
		Es gibt ein maximales Ideal $I_\lambda$ mit $(\lambda e-a)\A \subseteq I_\lambda$.
	\item \label{proof:Spektrum-von-a:I-abg}
		$I_\lambda \subsetneq \A$ ist abgeschlossen.
	\item \label{proof:Spektrum-von-a:AI-BA}
		$^\A/_{I_\lambda}$ ist eine Banachalgebra mit $\left({}^\A/_{I_\lambda}\right)^\times = {}^\A/_{I_\lambda} \backslash \{[0]\}$.
	\item \label{proof:Spektrum-von-a:Projektion}
		Die Projektion $f: \A \to ~^\A/_{I_\lambda} \cong \CC$ ist der gesuchte Homomorphismus.
\end{proofenum}

Zu \ref{proof:Spektrum-von-a:maxIdeal}: 
Wir definieren zunächst die Menge
	\[(\lambda e-a)\A := \{(\lambda\cdot e - a)\cdot b ~|~ b \in \A\} \subset \A. \] 
Diese ist offensichtlich ein Ideal. Ferner ist $\lambda \in \sigma_\A(a)$ und deswegen $\lambda e-a \notin \A^\times$. Damit ist $e \notin (\lambda e-a)\A$ (sonst wäre $e = (\lambda e - a)\cdot b$ für ein $b \in \A$ und damit $\lambda e-a \in \A^\times$). Also ist $(\lambda e-a)\A$ sogar ein echtes Ideal.

Betrachte nun die Menge aller echten Ideale, die $(\lambda e-a)\A$ enthalten:
	\[M := \{I \subsetneq A \text{ Ideal} ~|~ (\lambda e-a)\A \subseteq I\}\]
Zusammen mit der Inklusion $\subseteq$ als Ordnungsrelation ist dies eine partiell geordnete Menge. 

Falls diese Menge ein maximales Element besitzt, so wäre dieses eine passende Wahl für das gesuchte maximale Ideal $I_\lambda$. Dass ein solches maximales Element auch tatsächlich existiert, zeigen wir mit Hilfe des Lemmas von Zorn (\ref{satz:LZ}):

\Beh Sei $N \subseteq M$ eine nicht-leere\footnote{Für den Fall der leeren Teilmenge (die trivialerweise total geordnet ist), kann direkt $(\lambda e-a)\A$ selbst als obere Schranke gewählt werden.} total geordnete Teilmenge. Dann ist $J := \bigcup_{I \in N} I$ eine obere Schranke für $N$.
\begin{itemize}
	\item[$\bullet$] $J$ ist ein Ideal, denn:
	
	Sind $x,y \in J$, dann gibt es $I_x, I_y \in N$, sodass $x \in I_x, y \in I_y$. Da $N$ total geordnet ist bezüglich der Inklusion, gilt außerdem oBdA $I_x \subseteq I_y$. Also ist $x,y \in I_y$ und es folgt (da $I_y$ ein Ideal ist): $x+y \in I_y \subseteq J$.
	
	Außerdem gilt für alle $a \in \A$, dass $a\cdot y \in I_y \subseteq J$.
	
	\item[$\bullet$] $J$ ist ein echtes Ideal, denn:
	
	Angenommen es wäre $J = \A$. Dann wäre insbesondere $e \in J$ und damit $e \in I$ für ein $I \in N$. Da $I$ ein Ideal ist, wäre damit aber $a = a\cdot e \in I$ für jedes $a \in \A$.
	
	Das heißt es wäre $I = \A$ kein echtes Ideal, was im Widerspruch zur Definition von $N$ steht.
	
	\item[$\bullet$] $(\lambda e-a)\A$ ist in $J$ enthalten, denn:
	
	$N$ enthält wenigstens ein $I$ aus $M$. Dieses $I$ enthält nach Definition das Ideal $(\lambda e-a)\A$ und damit enthält auch $J$ dieses Ideal.
\end{itemize}
Also liegt $J$ in $M$ und ist offensichtlich eine obere Schranke für $N$, was die Behauptung zeigt.

Damit sind die Voraussetzung des Lemmas von Zorn erfüllt. Es folgt, dass $M$ ein maximales Element enthält, welches also gerade die in \ref{proof:Spektrum-von-a:maxIdeal} gewünschte Eigenschaft hat. Wir nennen es daher $I_\lambda$.

Zu \ref{proof:Spektrum-von-a:I-abg}:
Da Addition und Multiplikation in $\A$ nach \Cref{lemma:BAlg-Eigenschaften} stetig ist, ist auch $\overline{I_\lambda}$, der Abschluss von $I_\lambda$, ein Ideal. $\overline{I_\lambda}$ enthält aber mit $I_\lambda$ ein maximales Ideal und es                        muss folglich $\overline{I_\lambda} = I_\lambda$ oder $\overline{I_\lambda} = \A$ gelten.

Wäre $\overline{I_\lambda} = \A$, so läge $I_\lambda$ dicht in $\A$ und folglich wäre $I_\lambda \cap \{b \in \A ~|~ \norm{b-e} < 1\} \neq \emptyset$. Sei also $b$ aus $I_\lambda \cap \{b \in \A ~|~ \norm{b-e} < 1\}$, dann ist $\sum_{k=0}^\infty(e-b)^k$ absolut konvergent und das Inverse zu $b$, denn:
	\[b\sum_{k=0}^\infty(e-b)^k = (b-e)\sum_{k=0}^\infty(e-b)^k + e\sum_{k=0}^\infty(e-b)^k = -\sum_{k=0}^\infty(e-b)^{k+1} +\sum_{k=0}^\infty(e-b)^k = 1\]
Damit ist aber $\A = I_\lambda$, denn für beliebiges $c \in \A$ gilt:
	\[c = ce = cb\sum_{k=0}^\infty(e-b)^k = \left(c\sum_{k=0}^\infty(e-b)^k\right) b \in I_\lambda\]
Das jedoch steht im Widerspruch zu \ref{proof:Spektrum-von-a:maxIdeal} ($I_\lambda$ ist ein echtes Ideal). Daher muss $\overline{I_\lambda} = I_\lambda$ gelten und $I_\lambda$ ist tatsächlich abgeschlossen.


Zu \ref{proof:Spektrum-von-a:AI-BA}:
Da $I_\lambda$ also abgeschlossen ist, ist $^\A/_{I_\lambda}$ mit der Norm $\norm{[b]} := \underset{x \in I_\lambda}{\inf}\norm{b+x}$ nach \Cref{satz:Quotient} ein vollständiger normierter Vektorraum. Außerdem erhalten wir durch $[b]\cdot [c] := [bc]$ eine Multiplikation (mit Einselement $[e]$). Da $I_\lambda$ als Ideal auch unter Multiplikation abgeschlossen ist, ist diese Operation auf $^\A/_{I_\lambda}$ wohldefiniert. Ferner ist die Norm submultiplikativ, denn:
	\begin{align*}
		\norm{[b]} \cdot \norm{[c]} &= \underset{x \in I_\lambda}{\inf}\norm{b+x} \cdot \underset{y \in I_\lambda}{\inf}\norm{c+y} = \underset{x,y \in I_\lambda}{\inf}(\norm{b+x} \cdot \norm{c+y}) \geq \\
		&\geq \underset{x,y \in I_\lambda}{\inf}\norm{bc+by+cx+xy} \geq \underset{z \in I_\lambda}{\inf}\norm{bc+z} = \norm{[b]\cdot[c]}
	\end{align*}
Also ist $^\A/_{I_\lambda}$ eine Banachalgebra.

Noch zu zeigen ist $\left({}^\A/_{I_\lambda}\right)^\times = {}^\A/_{I_\lambda} \backslash \{[0]\}$. Hierbei folgend wir dem Beweis zu \cite[Lemma IX.2.5(e)]{Werner2011}.

Ist $[b] \in {}^\A/_{I_\lambda} \backslash \{[0]\}$, dann ist $b \notin I_\lambda$, denn sonst wäre $b-0 = b \in I_\lambda$, also $[b] = [0]$.

Definiere nun $J := \{cb + x ~|~ c\in\A, x\in I_\lambda\}$. Dies ist ein Ideal, welches $I_\lambda$ enthält (ist $x \in I_\lambda$, so ist $x = 0b+x \in J$). Wegen $b \in J \backslash I_\lambda$ ist $I_\lambda \subsetneq J$ und es folgt $J = \A$, da $I_\lambda$ nach \ref{proof:Spektrum-von-a:maxIdeal} ein maximales Ideal ist.

Insbesondere ist daher $e \in J$, es existieren also $c \in A, x \in I_\lambda$, sodass $cb + x = e$. Damit ist $[c] \in {}^\A/_{I_\lambda}$ das Inverse zu $[b]$:
	\[[c][b] = [cb] = [cb + x] = [e]\]

zu \ref{proof:Spektrum-von-a:Projektion}:
Ist $[b] \in {}^\A/_{I_\lambda}$, so gibt es nach \Cref{kor:spektrum-nicht-leer} ein $\lambda_b \in \sigma_{^\A/_{I_\lambda}}([b])$. Also ist $\lambda_b[e] - [b] \notin \left({}^\A/_{I_\lambda}\right)^\times$ und daher mit \ref{proof:Spektrum-von-a:AI-BA} bereits $\lambda_b[e] - [b] = [0]$. Dadurch ist $[b] = \lambda_b[e]$ und $\lambda_b \in \CC$ damit eindeutig bestimmt. Wir können daher die folgende Abbildung definieren:
	\[f: \A \to \CC: b \mapsto \lambda_b\]
Diese ist ein Algebrenhomomorphismus, denn für alle $b,c \in \A, \mu \in \CC$ gilt:
\begin{itemize}
	\item[$\bullet$] 
		$\lambda_{\mu b+c}[e] = [\mu b+c] = \mu[b]+[c] = \mu\lambda_b[e] + \mu\lambda_c[e] = (\mu\lambda_b +\lambda_c)[e]$
	
			$\qquad \Rightarrow f(\mu b + c) = \lambda_{\mu b+c} = \mu\lambda_b +\lambda_c  = \mu f(b) + f(c)$
	\item[$\bullet$]
		 $\lambda_{bc}[e] = [bc] = [b][c] = \lambda_b[e]\lambda_c[e] = \lambda_b\lambda_c[e]$
	
			$\qquad \Rightarrow f(bc) = \lambda_{bc} = \lambda_b\lambda_c  = f(b)f(c)$
	\item[$\bullet$]
		 $\lambda_e[e] = [e] = 1[e] ~ \Rightarrow ~ f(e) = \lambda_e = 1$
\end{itemize}
Zudem ist $f$ stetig, denn für $b \in \A$ und $\epsilon > 0$ gilt für alle $c \in \A$ mit $\norm{b-c} < \epsilon\norm{[e]}$:
	\begin{align*}
		|f(b)-f(c)| &= |\lambda_b - \lambda_c| = |\lambda_b - \lambda_c|\frac{\norm{[e]}}{\norm{[e]}} = \Big\lVert\lambda_b[e] - \lambda_c[e]\Big\rVert\frac{1}{\norm{[e]}}  = \\
					&= \norm{[b]-[c]}\frac{1}{\norm{[e]}} \leq \norm{b-c}\frac{1}{\norm{[e]}} < \epsilon\norm{[e]}\frac{1}{\norm{[e]}} = \epsilon
	\end{align*}	
Also ist $f \in \SpecC(\A)$. Weiterhin ist $\lambda e - a \in I_\lambda$, also $[0] = [\lambda e - a] \in {}^\A/_{I_\lambda}$ und daher $f(0) = f(\lambda e - a)$. Daraus folgt
	\[0 = f(0) = f(\lambda e - a) = \lambda f(e) - f(a) = \lambda - f(a)\]
und somit ist $f$ gerade der gesuchte Algebrenhomomorphismus mit $f(a) = \lambda$.

Da diese Konstruktion für beliebige $\lambda \in \sigma_\A(a)$ möglich ist, ergibt sich daraus die gesuchte Inklusion:
	\[\sigma_\A(a) \subseteq \SpecC(\A) \qedhere\]
\end{itemize}
\end{proof}

Hiermit haben wir nun die nötigen Voraussetzungen um zu zeigen, dass $\AlgIso$ isometrisch ist:

\begin{proof}[\Bew{$\AlgIso$ isometrisch}]
Sei $a \in \A$, dann ist $\{f(a) ~|~ f \in \SpecC(\A)\} = \sigma_\A(a)$ (\Cref{prop:Spektrum-von-a}) und es gilt:
	\[\underset{f \in \SpecC(\A)}{\sup}\left|f(a)\right| = \underset{\lambda \in \sigma_\A(a)}{\sup}\left|\lambda\right| = R_\A(a)\]
Aus \Cref{prop:R-gleich-Norm} wissen wir zudem, dass $R_\A(a) = \norm{a}$ ist, woraus insgesamt folgt, dass $\AlgIso$ isometrisch ist:
	\[\norm{\AlgIso(a)} = \underset{f \in \SpecC(\A)}{\sup}\left|f(a)\right| = R_\A(a) = \norm{a}.\]
\let\qed\relax
\end{proof}