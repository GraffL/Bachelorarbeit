\subsubsection{$\AlgIso$ ist bijektiv}\label{sec:Bijektiv}

Als letzte Eigenschaft bleibt noch zu zeigen, dass $\AlgIso$ auch bijektiv ist.

\begin{proof}[\Bew{$\AlgIso$ bijektiv}]\ 

\begin{itemize}
\item $\AlgIso$ ist injektiv, denn:

Sind $a, b \in \A$, sodass $\AlgIso(a) = \AlgIso(b)$, dann folgt (da $\AlgIso$ als Algebrenhomomorphismus insbesondere linear ist):
	\[0 = \norm{\AlgIso(a) - \AlgIso(b)} = \norm{\AlgIso(a-b)} = \norm{a-b}\]
und damit $a = b$.

\item $\AlgIso$ ist surjektiv, denn:

Wir wollen hier den Satz von Stone-Weierstraße (\ref{satz:SW}) anwenden. Zunächst ist leicht zu sehen, dass $\AlgIso(\A) \subseteq \stetig(\SpecC(\A))$ eine Unteralgebra ist ($\AlgIso$ ist ein Algebrenhomomorphismus und daher $\AlgIso(\A)$ abgeschlossen unter den Algebraoperationen). Ferner gilt:
\begin{itemize}
	\item In \Cref{sec:Algebrenhomomorphismus} haben wir bereits gezeigt, dass $\AlgIso(e)$ die konstante Einsabbildung ist (das Einselement in $\stetig(\SpecC(\A))$. Damit sind wegen $\AlgIso(\lambda e) = \lambda \AlgIso(e)$ für $\lambda \in \CC$ bereits alle konstanten Funktionen in $\AlgIso(\A)$ enthalten.
	\item Sind $f, g \in \SpecC(\A)$ mit $f \neq g$, so gibt es also ein $a \in \A$ mit $f(a) \neq g(a)$. Damit gilt:
		\[\AlgIso(a)(f) = \tau_a(f) = f(a) \neq g(a) = \tau_a(g) = \AlgIso(a)(g)\]
		Also ist $\tau_a \in \AlgIso(\A)$ eine Funktion, die $f$ und $g$ trennt.
	\item Ist $\tau_a \in \AlgIso(\A)$, so auch $\tau_{a^*}$, und es folgt nach \Cref{sec:CAlgHom}:
		\[\overline{\tau_a} = (\AlgIso(a))^* = \AlgIso(a^*) = \tau_{a^*} \in \AlgIso(\A)\]
\end{itemize}
Also liegt dem Satz von Stone-Weierstraß zufolge $\AlgIso(\A)$ dicht in $\stetig(\SpecC(\A))$, d.h. $\overline{\AlgIso(\A)} = \stetig(\SpecC(\A))$.

Gleichzeitig ist $\AlgIso(\A)$ aber abgeschlossen in $\stetig(\SpecC(\A))$. Ist nämlich $(\AlgIso(a_n))_{n\in\NN}$ eine Folge in $\AlgIso(\A)$, die in $\stetig(\SpecC(\A))$ konvergiert, so konvergiert auch die Folge $(a_n)_{n\in\NN}$ in $\A$ (da $\AlgIso$ isometrisch und ein Algebrenhomomorphismus ist). Aufgrund der Vollständigkeit von $\A$ hat diese Folge einen Grenzwert $a \in \A$ und $\AlgIso(a)$ ist - wieder aufgrund der Isometrie von $\AlgIso$ - der Grenzwert von $(\AlgIso(a_n))_{n\in\NN}$ in $\AlgIso(\A)$. 

Es ist folglich insgesamt
	\[\AlgIso(\A) = \overline{\AlgIso(\A)} = \stetig(\SpecC(\A))\]
und somit ist $\AlgIso$ surjektiv.
\end{itemize}

Hiermit ist der Satz von Gelfand-Neumark nun vollständig bewiesen. Die Abbildung $\AlgIso: \A \to \stetig(\SpecC(\A)): a \mapsto \left(\tau_a: f \mapsto f\left(a\right)\right)$ ist also tatsächlich ein isometrischer Isomorphismus von \CAlgn. Er zeigt damit auch, dass tatsächlich jede \CAlg{} bis auf isometrische Isomorphie eine \CAlg{} der stetigen komplexwertigen Funktionen über einem kompakten Hausdorffraum ist.
\end{proof}