{\crefname{prop}{Prop.}{Prop.}
\crefname{section}{Absch.}{Absch.}
\crefname{kor}{Kor.}{Kor.}
\tikzstyle{Def} = [rectangle, draw, fill=gray!50, 
    text width=4.5em, text badly centered]
\tikzstyle{Prop} = [rectangle, draw, 
    text centered, rounded corners]
\tikzstyle{Absch} = [rectangle, draw, 
    text centered]
\tikzstyle{keinBeweis} = [rectangle, fill=gray!35, 
    text centered]    
\tikzstyle{Text} = [ 
    text centered]
\tikzstyle{BewTeil} = []

\tikzstyle{line} = [draw, -latex']
\tikzstyle{line2} = [draw]

\begin{tikzpicture}[node distance = 6em, auto]

\node [Text] (satz) {\large $\AlgIso: \A \to \stetig(\SpecC(\A)): a \mapsto \left(\tau_a:f \mapsto f(a)\right)$};
	\node [BewTeil, left of=satz, node distance=40em] (legende) {\large Legende:};
		\node [keinBeweis, below of=legende, node distance=2em, text width=1em, text height=1em] (grau) {};
		\node [BewTeil, right of=grau, node distance=10em, text width=17em] (grau-erk) {= ohne Beweis verwendete Aussage};
		\node [Absch, below of=grau, node distance=2em, text width=1em, text height=1em] (viereck) {};
		\node [BewTeil, right of=viereck, node distance=10em, text width=17em] (viereck-erk) {= wesentlicher Beweisschritt};
		\node [Prop, below of=viereck, node distance=2em, text width=1em, text height=1em] (rund) {};
		\node [BewTeil, right of=rund, node distance=10em, text width=17em] (rund-erk) {= Hilfsaussage};
	\node [below of=satz, node distance=3em] (hilf-satz) {};
    
    %Bijektiv-Block
    \node [Absch, left of=hilf-satz, node distance=16em, text width=8em] (bi) {\Cref{sec:Bijektiv} \\ $\AlgIso$ bijektiv};    
    	\node [below of=bi](hilf-bi) {}; 
    	\node [Text, left of=hilf-bi, node distance=8em] (inj) {$\AlgIso$ injektiv};
    			\node [left of=inj, node distance=18em] (links-inj) {};
       		%Isomometrisch
    		\node [Absch, below of=inj, text width=8em] (iso) {\Cref{sec:isometrisch} \\ $\AlgIso$ isometrisch};
				\node [below of=iso](hilf-iso) {};     
    			\node [Prop, left of=hilf-iso, node distance=13em, text width=8em] (RAnorm) 
    			{\Cref{prop:R-gleich-Norm} \\ $R_\A(a) = \norm{a}$};
    					\node [below of=RAnorm, node distance=3em, text width=4em] (unter-RAnorm) {};
    				\node [Text, below of=RAnorm] (hilf-RAnorm) {};    				
    				\node [Prop, below of=hilf-RAnorm, text width=8em] (RAlima) 
    				{\Cref{prop:R-groesser-lima} \\ in Banachalg.: \\ $R\!_\A(a)\!\!\geq\!\!\lim\!\norm{\!a^k\!}^\frac{1}{k}$}; 
    				\node [Prop, below of=RAlima, text width=8em] (konv)
    				{\Cref{prop:Konvergenz} \\ $\alpha_n \to \epsilon < 1$ \\ $\Rightarrow (\alpha_n)^n \to 0$};
       			\node [Prop, right of=hilf-iso, node distance=10em, text width=15em] (fasigma) 
       			{\Cref{prop:Spektrum-von-a} \\ $\{f(a)|f\in\SpecC(\A)\}=\sigma_\A(a)$};
					\node [BewTeil, below of=fasigma] (bew-fasigma) 
					{\qquad für $\lambda \in \sigma_\A(a)$:};
					\node [BewTeil, below of=bew-fasigma, text width=10em, node distance=1.5em] (bew-fasigma1) 
					{\ref{proof:Spektrum-von-a:maxIdeal} $(\lambda e-a)\A \subseteq I_\lambda$ \\ \qquad\!\!\!\!\! max. Ideal};
						\node [keinBeweis, left of=bew-fasigma1, node distance=12em] (zorn) 
						{Lemma v. Zorn (\ref{satz:LZ})};						
					\node [BewTeil, below of=bew-fasigma1, text width=10em, node distance=2.5em] (bew-fasigma2) 
					{\ref{proof:Spektrum-von-a:I-abg} $I_\lambda$ abg.};										    		
							\node [left of=bew-fasigma2, node distance=6.5em] (links-bew-fasigma2) {};
					\node [BewTeil, below of=bew-fasigma2, text width=10em, node distance=2em] (bew-fasigma3) 
					{\ref{proof:Spektrum-von-a:AI-BA} $^\A/_{I_\lambda}$ BA \& Körper};	
						\node [keinBeweis, left of=bew-fasigma3, node distance=12em] (quotient) 
						{Quotientenalg. (\ref{satz:Quotient})};						
					\node [BewTeil, below of=bew-fasigma3, text width=10em, node distance=2em] (bew-fasigma4) 
					{\ref{proof:Spektrum-von-a:Projektion} $f:\A \to {}^\A/_{I_\lambda} \cong \CC$};
						\node [Prop, left of=bew-fasigma4, node distance=12em, text width=8em] (nichtleer) 
						{\Cref{kor:spektrum-nicht-leer}:\\ $\sigma_\A(a) \neq \emptyset$};						
				    
    	\node [Text, right of=hilf-bi, node distance=8em] (sur) {$\AlgIso$ surjektiv};
			\node [below of=sur](hilf-sur) {};     
    		\node [Text, left of=hilf-sur, node distance=5em] (HAabg) {$\AlgIso(\A)$ abg.};
    		\node [Text, right of=hilf-sur, node distance=5em] (HAdicht) {$\AlgIso(\A)$ dicht};
				\node [below of=HAdicht] (hilf-HAdicht) {};
					\node [keinBeweis, right of=hilf-HAdicht, text width=7em, node distance=4.3em] (SW) 
					{Satz v. Stone-Weierstraß (\ref{satz:SW})};   
				\node [BewTeil, below of=hilf-HAdicht] (bew-HAdicht) {$\SpecC(\A) \dots$};
				\node [BewTeil, below of=bew-HAdicht, text width=8.4em, node distance=2em] (bew-HAdicht1) 
				{$\bullet$ ist kompakt};
						\node [right of=bew-HAdicht1, node distance=7em] (rechts-bew-HAdicht1) {};	
				\node [BewTeil, below of=bew-HAdicht1, text width=8.4em, node distance=2em] (bew-HAdicht2) 
				{$\bullet$ enth. konst. Fkt.};
						\node [right of=bew-HAdicht2, node distance=5em] (rechts-bew-HAdicht2) {};
				\node [BewTeil, below of=bew-HAdicht2, text width=8.4em, node distance=2em] (bew-HAdicht3) 
				{$\bullet$ trennt Pkte.};	
				\node [BewTeil, below of=bew-HAdicht3, text width=8.4em, node distance=2em] (bew-HAdicht4) 
				{$\bullet$ ist selbstadj.};	

	% Sternhom-Block
    \node [Absch, right of=hilf-satz, node distance=11em, text width=10em] (sternhom) 
    {\Cref{sec:CAlgHom} \\ $\AlgIso$ \CAlgHom}; 
    		\node [right of=sternhom] (rechts-sternhom) {}; 
    	\node [Text, below of=sternhom] (stern) {$\AlgIso(a^*) = (\AlgIso(a))^*$};    
    			\node [left of=stern, node distance=5.5em] (links-stern) {}; 
      	\node [Prop, below of=stern, text width=8em] (sternkonj) 
    	{\Cref{prop:Stern-zu-Konjugation} \\ für $f: \A \to \CC$ \\ Alghom.: \\ $f(a^*) = \overline{f(a)}$};
    	\node [Prop, below of=sternkonj, text width=8em] (selbstadj) 
    	{\Cref{prop:Spektrum-reell} \\ für $a$ selbstadj.: \\ $\sigma_\A(a) \subseteq \RR$};
    			\node [below of=selbstadj, node distance=3em, text width=4em] (unter-selbstadj) {};    	
    	\node [Prop, below of=selbstadj, text width=8em] (eEig) 
    	{\Cref{lemma:CAlg-Eigenschaften} \\ $\norm{e} = 1$, $e^* = e$};
    	%Lemma 1.15 (Spec komp.)
    	\node [Absch, below of=eEig, text width=8em] (Speckomp) {\Cref{lemma:MA} \\ $\SpecC(\A)$ komp. HD-Raum}; 
    	\node [keinBeweis, below of=Speckomp, text width=8em] (alaoglu) {Satz v. Banach-Alaoglu (\ref{satz:BA})};     	
    	
    %Alghom	
	\node [Absch, below of=bew-HAdicht4, text width=15em, node distance=6.6em] (Alghom) 
	{\Cref{sec:Algebrenhomomorphismus} \\ $\AlgIso$ Algebrenhomomorphismus};    	

	%stetig
	\node [Absch, below of=nichtleer, text width=8em, node distance=4em] (stetig) 
	{\Cref{lemma:BAlg-Eigenschaften} \\ $+$, $\cdot$, $^{-1}$ stetig}; 


%Pfade:
	%Zu bijektiv:
	\path [line] (inj) -- (bi);
		\path [line] (iso) -- (inj);
			\path [line] (RAnorm) -- (iso);
				\path [line] (RAlima) -- (RAnorm);
					\path [line] (konv) -- (RAlima);
			\path [line] (fasigma) -- (iso);
				\path [line] (bew-fasigma) -- (fasigma);
					\path [line] (zorn) -- (bew-fasigma1);
					\path [line] (quotient) -- (bew-fasigma3);
					\path [line] (nichtleer) -- (bew-fasigma4);
						\path [line] (RAlima.east) -- (nichtleer.west);
	\path [line] (sur) -- (bi);
		\path [line] (HAabg) -- (sur);
			\path [line] (iso) -- (HAabg);
		\path [line] (HAdicht) -- (sur);
			\path [line] (bew-HAdicht) -- (HAdicht);
				\path [line2] (Speckomp) -| (rechts-bew-HAdicht1.center);
					\path [line] (rechts-bew-HAdicht1.center) -- (bew-HAdicht1);
					\path [line] (alaoglu) -- (Speckomp);
	%Zu Sternhom:
	\path [line] (stern) -- (sternhom);
	\path [line] (sternkonj) -- (stern);
		\path [line] (fasigma) -- (sternkonj.west);
	\path [line] (selbstadj) -- (sternkonj);
	\path [line] (eEig) -- (selbstadj);	
	
    %Von Algebrenhomomorphismus:
    \path [line2] (Alghom.east) -| (rechts-sternhom.east);
        \path [line] (rechts-sternhom.east) -- (sternhom);
    \path [line2] (Alghom.north -| rechts-bew-HAdicht2.east) -- (rechts-bew-HAdicht2.east);
    	\path [line] (rechts-bew-HAdicht2.east) -- (bew-HAdicht2);
    \path [line2] (Alghom.west) -| (links-inj.west);
	    \path [line] (links-inj.west) -- (inj);
	\path [line2] (Alghom.north -| hilf-sur) -- (fasigma.south -| hilf-sur);
		\path [line] (fasigma.north -| hilf-sur) |- (HAabg);
	
	%Von stetig:
	\path [line] (stetig.west) -- (RAlima);
	\path [line2] (stetig.east) -| (links-bew-fasigma2.center);
		\path [line] (links-bew-fasigma2.center) -- (bew-fasigma2);
	    
	%Von stern zu selbstadj    
    \path [line2] (stern.west) -- (links-stern.east);
	    \path [line] (links-stern.east) |- (bew-HAdicht4);
	    
	%Von SW:
	\path [line2] (SW) -- (hilf-HAdicht.center);	
	
	%RAlima nach selbstadj
	\path [line2] (RAnorm.south -| unter-RAnorm.east) -- (unter-RAnorm.east);
		\path [line2] (unter-RAnorm) -- (unter-selbstadj);
		\path [line] (unter-selbstadj.west) -- (selbstadj.south -| unter-selbstadj.west);    
	          
\end{tikzpicture}
}